\documentclass[a4paper,12pt]{article}
% math symbols
\usepackage{amssymb,amsmath}
\synctex=1
% for different compilers
\usepackage{ifpdf}
% geometry of page
\usepackage[margin=2.1cm]{geometry}

% if pdflatex, then
\ifpdf
\usepackage[russian]{babel}
\usepackage[utf8]{inputenc}
\usepackage[unicode]{hyperref}
\usepackage[pdftex]{graphicx}
\usepackage{cmlgc}
% if xelatex, then
\else
% math fonts
\usepackage{fouriernc}
% xelatex specific packages
\usepackage[xetex]{hyperref}
\usepackage{xltxtra}	% \XeLaTeX macro
\usepackage{xunicode}	% some extra unicode support
\defaultfontfeatures{Mapping=tex-text}
\usepackage{polyglossia}	% instead of babel in xelatex
\usepackage{indentfirst}	% 
\setdefaultlanguage{russian}
% fonts
\newfontfamily\cyrillicfont{SchoolBookC}
\newfontfamily\cyrillicfontsf{TextBookC}
\setmonofont{Consolas}
\fi

% several pictures in one figure
\usepackage{subfig}
% calc in TeX expressions
\usepackage{calc}
% nice pictures and plots
\usepackage{pgfplots,tikz,circuitikz}
% different libraries for pictures
\usetikzlibrary{%
  arrows,%
  calc,%
  patterns,%
  decorations.pathreplacing,%
  decorations.pathmorphing,%
  decorations.markings,%
  decorations.text%
}

\usepackage{tkz-euclide}

\graphicspath{{./pics/}}

% colors of the hyperlinks
\hypersetup{colorlinks,%
  citecolor=blue,%
  urlcolor=blue,%
  linkcolor=red
}

\newcommand{\foot}[1]{\mbox{\footnotesize{#1}}}

% счётчик задач
\newcounter{notask}
\setcounter{notask}{1}

% условие без картинки
\newcommand{\task}[1]{
  \hrule
  \hbox to \textwidth {%
    \vrule
    \parbox[t]{0.04\textwidth}{\smallskip \centering \arabic{notask}}%
    \vrule%
    \hfill%
    \parbox[t]{0.93\textwidth}{\smallskip #1 \smallskip}\hfill%
    \vrule
  }
  \hrule
  \addtocounter{notask}{1}
  \pagebreak[2]
}

\newlength{\h}
\newsavebox{\taskbox}
\newlength{\x}
\newsavebox{\pictbox}

% условие с картинкой (картинка выравнивается по центру)
\newcommand{\taskpic}[2]{
  \savebox{\taskbox}{\parbox[t]{0.93\textwidth-4.3cm}{\smallskip #1 \smallskip}}
  \savebox{\pictbox}{\parbox[t]{4cm}{\smallskip \centering
      \vspace{0pt} #2 \smallskip}}
  \h=\ht\taskbox
  \advance\h\dp\taskbox
  \x=\ht\pictbox
  \advance\x\dp\pictbox
  \hrule
  \hbox to \textwidth {%
    \vrule\parbox[t][\maxof{\h}{\x}][t]{0.04\textwidth}{ \smallskip
      \centering \arabic{notask} }\vrule%
    \hfill\parbox[t][\maxof{\h}{\x}][t]{0.93\textwidth-4.3cm}{\smallskip #1
      \smallskip}\hfill\vrule%
    \hfill\parbox[t][\maxof{\h}{\x}][c]{4cm}{\hfil #2 \hfil}\hfill\vrule
  }
  \hrule
  \addtocounter{notask}{1}
  \pagebreak[2]
}

% протокол физбоя
% ширина протокола, задаётся вручную, по умолчанию равен ширине текста
% на странице
\newlength{\pwidth}
\pwidth=\textwidth
% высота ячейки
\newlength{\pheightcell}
\pheightcell=1.75cm
% баллы жюри
\newcounter{jury}
\setcounter{jury}{12}
% номер вызова
\newcounter{nocall}
\setcounter{nocall}{1}
% заголовок протокола: названия команд
\newcommand{\pdtitle}[2]{
  \draw[thick] (0,0) rectangle (0.1*\pwidth,-\pheightcell) node[midway] {\textbf{№}};
  \draw[thick] (0.1*\pwidth,0) rectangle (0.46*\pwidth,-\pheightcell)
  node[midway] {\textbf{#1}};
  \draw[thick] (0.46*\pwidth,0) rectangle (0.82*\pwidth,-\pheightcell)
  node[midway] {\textbf{#2}}; 
  \draw[thick] (0.82*\pwidth,0) rectangle (\pwidth,-\pheightcell)
  node[midway] {\textbf{Ж}};
}

% вызов на задачу: порядок вызова, номер задачи, баллы
% отвечающего и оппонента, инициалы участников
% условные обозначения для порядка вызова:
% -> --- команда 1 вызывает команду 2
% <- --- команда 2 вызывает команду 1
% ->| --- команда 1 вызывала команду 2, но получила проверку корректности
% |<- --- команда 2 вызывала команду 1, но получила проверку корректности
\newcommand{\pdcall}[6]{
  % определяем местонахождение очередной ячейки
  \def\pos{\arabic{nocall}*\pheightcell};
  \def\mid{-\pos-0.5*\pheightcell};
  % места, где будут стоять баллы
  \coordinate (bone) at (0.28*\pwidth,\mid);
  \coordinate (btwo) at (0.64*\pwidth,\mid);
  % рамка вокруг ячейки
  \draw[thick] (0,-\pos) rectangle (\pwidth,-\pos-\pheightcell);
  % номер задачи и рамка
  \draw[thick] (0,-\pos) rectangle (0.1*\pwidth,-\pos-\pheightcell)
  node[midway] {#2};
  % баллы слева направо
  \node (a) at (bone) {\Large{#3}};
  \node (b) at (btwo) {\Large{#4}};
  % инициалы отвечающего и оппонента
  \node at ([shift={(1.1cm,0.4cm)}]a) {#5};
  \node at ([shift={(-1.1cm,0.4cm)}]b) {#6};
  % switch
  \def\calllr{->}
  \def\callrl{<-}
  \def\calllor{->|}
  \def\callrol{|<-}
  \begingroup
  \def\tmp{#1}%
  \ifx\tmp\calllr
  \draw[very thick,*->] (a.east) -- (b.west);
  \else\ifx\tmp\callrl
  \draw[very thick,<-*] (a.east) -- (b.west);
  \else\ifx\tmp\calllor
  \draw[very thick,*->,rounded corners=0.2cm] (a.east) -- (b.west)
  -- ++(0,-0.4cm) -- ($(a.east)+(0,-0.4cm)$);
  \else\ifx\tmp\callrol
  \draw[very thick,*->,rounded corners=0.2cm] (b.west) -- (a.east)
  -- ++(0,-0.4cm) -- ($(b.west)+(0,-0.4cm)$);
  \fi\fi\fi\fi
  \endgroup
  % баллы жюри и рамка
  \addtocounter{jury}{-#3}
  \addtocounter{jury}{-#4}
  \draw[thick] (0.82*\pwidth,-\pos) rectangle (\pwidth,-\pos-\pheightcell)
  node[midway] {\Large{\arabic{jury}}};
  % переход на следующий круг
  \setcounter{jury}{12}
  \addtocounter{nocall}{1}
}

% подведение итогов
% \psum{баллы команды 1}{баллы команды 2}{баллы жюри}
\newcommand{\pdsum}[3]{ 
  \def\pos{\arabic{nocall}*\pheightcell-1cm};
  % рамка вокруг ячейки
  \draw[thick] (0,-\pos) rectangle (\pwidth,-\pos-\pheightcell);
  % "Итог" и рамка
  \draw[thick] (0,-\pos) rectangle (0.1*\pwidth,-\pos-\pheightcell)
  node[midway] {Итог:};
  \draw[thick] (0.1*\pwidth,-\pos) rectangle
  (0.46*\pwidth,-\pos-\pheightcell) node[midway] {\LARGE{#1}};
  
  \draw[thick] (0.46*\pwidth,-\pos) rectangle (0.82*\pwidth,-\pos-\pheightcell)
  node[midway] {\LARGE{#2}}; 
  
  \draw[thick] (0.82*\pwidth,-\pos) rectangle (\pwidth,-\pos-\pheightcell)
  node[midway] {\LARGE{#3}}; 
  \setcounter{nocall}{1}
}


% тройной физбой
% заголовок
\newcommand{\pttitle}[3]{
  \draw[thick] (0,0) rectangle (0.1*\pwidth,-\pheightcell) node[midway] {\textbf{№}};
  \draw[thick] (0.1*\pwidth,0) rectangle
  ++(0.26*\pwidth,-\pheightcell) node[midway] {\textbf{#1}};
  \draw[thick] (0.36*\pwidth,0) rectangle ++(0.26*\pwidth,-\pheightcell)
  node[midway] {\textbf{#2}}; 
  \draw[thick] (0.62*\pwidth,0) rectangle ++(0.26*\pwidth,-\pheightcell)
  node[midway] {\textbf{#2}}; 
  \draw[thick] (0.88*\pwidth,0) rectangle (\pwidth,-\pheightcell)
  node[midway] {\textbf{Ж}};
}

% вызов 
% первый аргумент показывает, есть ли проверка корректности
% если вызов обычный - аргумент пустой
% если проверка - можно написать всё, что угодно (например,
% какую-нибудь букву)
% \ptcall{#1}{номер
% задачи}{баллы1}{баллы2}{баллы3}{ф.и.1}{ф.и.2}{ф.и.3}
\newcommand{\ptcall}[8]{
  \edef\cl{\arabic{nocall}};
  % определяем местонахождение очередной ячейки
  \def\pos{\arabic{nocall}*\pheightcell};
  \def\mid{-\pos-0.5*\pheightcell};
  % места, где будут стоять баллы
  \coordinate (bone) at (0.23*\pwidth,\mid);
  \coordinate (btwo) at (0.49*\pwidth,\mid);
  \coordinate (bthree) at (0.75*\pwidth,\mid);
  % рамка вокруг ячейки
  \draw[thick] (0,-\pos) rectangle (\pwidth,-\pos-\pheightcell);
  % номер задачи и рамка
  \draw[thick] (0,-\pos) rectangle (0.1*\pwidth,-\pos-\pheightcell)
  node[midway] {#2};
  % баллы слева направо
  \node (a) at (bone) {\Large{#3}};
  \node (b) at (btwo) {\Large{#4}};
  \node (c) at (bthree) {\Large{#5}};
  % инициалы слева направо
  \node at ([shift={(0,0.6cm)}]a) {#6};
  \node at ([shift={(0,0.6cm)}]b) {#7};
  \node at ([shift={(0,0.6cm)}]c) {#8};
  % поскольку порядок вызовов определяется автоматически,
  % здесь сразу рисуется стрелка вызова в зависимости от номера хода
  % кроме того, проверяем, есть ли проверка корректности
  % 1-й ход
  \ifnumequal{\cl}{1}{
    \ifblank{#1}
    {
      \draw[very thick,*->] (a.east) -- (b.west);
    }
    {
      \draw[very thick,*->,rounded corners=0.2cm]
      (a.east) -- (b.west) -- ++(0,-0.4) -- ($(a.east)+(0,-0.4cm)$);
    }
  }{}
  % 2-й ход
  \ifnumequal{\cl}{2}{
    \ifblank{#1}
    {
      \draw[very thick,*->] (b.east) -- (c.west);
    }
    {
      \draw[very thick,*->,rounded corners=0.2cm] (b.east) -- (c.west)
      -- ++(0,-0.4cm) -- ($(b.east)+(0,-0.4cm)$);
    }
  }{}
  % 3-й ход
  \ifnumequal{\cl}{3}{
    \draw[very thick,*-] (c.east) -- (0.88*\pwidth,\mid);
    \ifblank{#1}
    {
      \draw[very thick,->] (0.1*\pwidth,\mid) -- (a.west);
    }
    {
      \draw[very thick,rounded corners=0.2cm] (0.1*\pwidth,\mid) --
      (a.west) -- ++(0,-0.4cm) -- (0.1*\pwidth,\mid-0.4cm);
      \draw[very thick,->] (0.88*\pwidth,\mid-0.4cm) -- ($(c.east) +
      (0,-0.4cm)$);
    }
  }{}
  % 4-й ход
  \ifnumequal{\cl}{4}{
    \draw[very thick,-*] (0.1*\pwidth,\mid) -- (a.west);
    \ifblank{#1}
    {
      \draw[very thick,->] (0.88*\pwidth,\mid) -- (c.east);
    }
    {
      \draw[very thick,rounded corners=0.2cm] (0.88*\pwidth,\mid) --
      (c.east) -- ++(0,-0.4cm) -- (0.88*\pwidth,\mid-0.4cm);
      \draw[very thick,->] (0.1*\pwidth,\mid-0.4cm) --
      ($(a.west)+(0,-0.4cm)$); 
    }
  }{}
  % 5-й ход
  \ifnumequal{\cl}{5}{
    \ifblank{#1}
    {
      \draw[very thick,*->] (c.west) -- (b.east);
    }
    {
      \draw[very thick,rounded corners=0.2cm,*->] (c.west) -- (b.east)
      -- ++(0,-0.4cm) -- ($(c.west)+(0,-0.4cm)$);
    }
  }{}
  % 6-й ход
  \ifnumequal{\cl}{6}{
    \ifblank{#1}
    {
      \draw[very thick,*->] (b.west) -- (a.east);
    }
    {
      \draw[very thick,rounded corners=0.2cm,*->] (b.west) -- (a.east)
      -- ++(0,-0.4cm) -- ($(b.west)+(0,-0.4cm)$) ;
    }
  }{}
  % баллы жюри и рамка
  \addtocounter{jury}{-#3} 
  \addtocounter{jury}{-#4} 
  \addtocounter{jury}{-#5} 
  \draw[thick] (0.88*\pwidth,-\pos) rectangle
  (\pwidth,-\pos-\pheightcell) node[midway] {\Large{\arabic{jury}}};
  % переход на следующий круг
  \setcounter{jury}{12}
  \addtocounter{nocall}{1}
}

% подведение итогов
\newcommand{\ptsum}[4]{ 
  \def\pos{\arabic{nocall}*\pheightcell-1cm};
  % рамка вокруг ячейки
  \draw[thick] (0,-\pos) rectangle (\pwidth,-\pos-\pheightcell);
  % "Итог" и рамка
  \draw[thick] (0,-\pos) rectangle (0.1*\pwidth,-\pos-\pheightcell)
  node[midway] {Итог:};
  % баллы
  \draw[thick] (0.1*\pwidth,-\pos) rectangle
  (0.36*\pwidth,-\pos-\pheightcell) node[midway] {\LARGE{#1}};
  
  \draw[thick] (0.36*\pwidth,-\pos) rectangle (0.62*\pwidth,-\pos-\pheightcell)
  node[midway] {\LARGE{#2}}; 
  
  \draw[thick] (0.62*\pwidth,-\pos) rectangle (0.88*\pwidth,-\pos-\pheightcell)
  node[midway] {\LARGE{#3}}; 
  
  \draw[thick] (0.88*\pwidth,-\pos) rectangle (\pwidth,-\pos-\pheightcell)
  node[midway] {\LARGE{#4}}; 
  \setcounter{nocall}{1}
}


\tikzset{>=latex,%
  interface/.style={postaction={draw,decorate,decoration={border,angle=45,amplitude=0.2cm,segment
        length=1.4323mm}}},%
  spring/.style={decorate,decoration={snake,amplitude=1mm, segment
      length=2mm},thick},%
  marrow/.style={postaction={draw,decorate,decoration={markings,
        mark=at position 0.6 with {\arrow{latex}}}}}}

\pgfplotsset{compat=newest}

\begin{document}
%\parindent=0mm

\begin{center}
  \LARGE{Работа, псевдоработа и теплота}\\[0.2cm]
  \large{Игорь Шендерович}\\[0.2cm]
  \normalsize{\texttt{shender.i@gmail.com}}
\end{center}

\begin{abstract}
  В данной заметке рассматривается взаимосвязь между совершённой
  работой и теплотой. Показано, что обычное интегрирование второго
  закона Ньютона не даёт описания тепловой стороны механических
  явлений; для их описания применяется первый закон термодинамики. С
  его помощью удаётся установить, что работа силы трения в некоторых
  случаях оказывается отличной от произведения силы трения на
  перемещение предмета. 
\end{abstract}

\section{Теорема о движении центра масс и псевдоработа.}
\label{sec:center_of_mass_energy_work}

Как известно, движение центра масс системы описывается уравнением

\begin{equation}
  \label{eq:c_o_m_newton}
  M \vec{a}_{\foot{ц.м.}} = \sum \vec{F}^{\foot{ext}},
\end{equation}
где $M$ --- общая масса системы, $\vec{a}_{\foot{ц.м.}}$ --- ускорение
центра масс, $\sum \vec{F}^{\foot{ext}}$ --- сумма всех
\textit{внешних} сил, действующих на систему. 

Из этого равенства можно получить что-то похожее на закон сохранения
энергии. Действительно, домножим правую и левую части на
$d\vec{r}_{\foot{ц.м.}}$ --- радиус--вектор перемещения центра масс. 

\begin{equation}
  \label{eq:c_o_m_to_pseudowork_I}
  M \frac{d \vec{v}_{\foot{ц.м.}}}{dt} d\vec{r}_{\foot{ц.м.}} = \sum
  \vec{F}^{\foot{ext}} d\vec{r}_{\foot{ц.м.}} \quad
  \Longleftrightarrow \quad M
  \vec{v}_{\foot{ц.м.}} d\vec{v}_{\foot{ц.м.}} = \sum
  \vec{F}^{\foot{ext}} d\vec{r}_{\foot{ц.м.}} 
\end{equation}
Проинтегрируем последнее равенство. Слева стоит дифференциал от
$\vec{v}^2_{\foot{ц.м.}}/2$, поэтому при интегрировании получим
\begin{equation}
  \label{eq:c_o_m_to_pseudowork_II}
  \frac{M \Delta v^2_{\foot{ц.м.}}}{2} = \int \sum \vec{F}^{\foot{ext}} d\vec{r}_{\foot{ц.м.}} 
\end{equation}

Выглядит действительно как закон сохранения энергии --- слева
изменение кинетической энергии центра масс, справа --- величина,
напоминающая суммарную работу внешних сил. Однако, это не совсем
работа --- как можно заметить, здесь сила умножается не на перемещение
точки приложения силы (как должно быть), а на перемещение центра масс.

Легко видеть, что это разные величины. Рассмотрим, например, процесс
сжатия шарика, наполненного газом, с помощью внешних сил. Если мы
сжимаем шарик равномерно со всех сторон, то его центр масс
перемещаться не будет (и сумма внешних сил будет равна нулю). При
этом, как мы знаем из термодинамики, совершённая работа будет даваться
выражением $A = \int  p dV$, в то время как в правой части формулы
\eqref{eq:c_o_m_to_pseudowork_II} получается ноль. Будем поэтому
называть объект в правой части уравнения
\eqref{eq:c_o_m_to_pseudowork_II} \textit{псевдоработой}. 

Рассмотрим несколько простых примеров.

\begin{figure}[h]
  \centering
  \begin{tikzpicture}
    \draw[interface] (5.2,0) -- (0.2,0);
    \draw[fill=gray!40,very thick] (2,0) rectangle (4,1);
    \draw[very thick,blue,->] (4.5,0.5) -- (5,0.5) node[above] {$v_0$};
    \draw[very thick,blue,->] (2,0) -- (1,0) node[above]
    {$F_{\foot{тр}}$};
  \end{tikzpicture}
  \caption{Останавливающийся брусок. }
  \label{fig:block_friction_I}
\end{figure}

Брусок массой $m$ со начальной скоростью $v_0$ въезжает на шероховатый стол,
коэффициент трения между ним и столом равен $\mu$
(рис. \ref{fig:block_friction_I}). Какой путь $S$ пройдёт
брусок перед тем, как остановиться?

Здесь всё очевидно. Изменение кинетической энергии центра масс равно
$mv_0^2/2$, по формуле \eqref{eq:c_o_m_to_pseudowork_II} оно равно
произведению внешней силы (которая в данном случае одна: сила трения)
на перемещение центра масс, то есть, на интересующее нас перемещение
$S$. Получаем
\begin{equation}
  \label{eq:problem_1_easy}
  \frac{mv_0^2}{2} = \mu m g S \quad \Longrightarrow \quad S = \frac{v_0^2}{2\mu g}. 
\end{equation}

Конечно, этот же ответ можно получить другим способом: поняв, что
ускорение центра масс бруска постоянно и равно $a_{\foot{ц.м.}} = -\mu
g$, написать уравнение для скорости $v_{\foot{ц.м.}} = v_0 - \mu g
t$. Отсюда получается, что время до остановки равно $T = v_0/\mu g$,
ну а пройденное расстояние, в самом деле, даётся формулой $S = v_0^2/2
\mu g$. 

Второй пример тоже не отличается сложностью. Представим себе такой же
брусок, но теперь его тянут по столу с постоянной силой $F$, причём
тянут так, что он двигается с постоянной скоростью
(рис. \ref{fig:block_friction_II}). Что в этом случае
даст нам уравнение \eqref{eq:c_o_m_to_pseudowork_II}?

\begin{figure}[h]
  \centering
  \begin{tikzpicture}
    \draw[interface] (5.2,0) -- (0.2,0);
    \draw[fill=gray!40,very thick] (2,0) rectangle (4,1);
    \draw[very thick,blue,->] (4,0.5) -- (5,0.5) node[above] {$F$};
    \draw[very thick,blue,->] (2,0) -- (1,0) node[above]
    {$F_{\foot{тр}}$};
  \end{tikzpicture}
  \caption{Брусок, двигающийся с постоянной скоростью. }
  \label{fig:block_friction_II}
\end{figure}

Изменение кинетической энергии, очевидно, будет равно нулю: скорость
бруска не меняется. Внешних сил тут уже две: сила $F$, с которой тянут
брусок, и равная ей по модулю (и противоположная по направлению) сила
$F_{\foot{тр}}$. Соответственно, суммарная \textit{псевдоработа} этих сил равна
нулю. Уравнение \eqref{eq:c_o_m_to_pseudowork_II} даёт нам $0=0$.

Всё это вроде бы неплохо, уравнения не противоречат друг другу, но
есть одна тонкость: все мы знаем, что при движении по шероховатой
поверхности брусок и стол должны нагреться. Где же в наших уравнениях
эта выделяющаяся теплота? Она не появляется именно потому, что это не
совсем закон сохранения энергии, а проинтегрированный второй закон
Ньютона с псевдоработой в правой части. 

Из-за псевдоработы в правой части уравнение
\eqref{eq:c_o_m_to_pseudowork_II} иногда ошибочно атрибутируется как
закон сохранения энергии. Это приводит к непониманию сути некоторых
физических явлений. В следующих разделах мы постараемся разобраться с
этим. 

\section{Первый закон термодинамики.}
\label{sec:flt}

Разумеется, справедливость уравнения \eqref{eq:c_o_m_to_pseudowork_II}
не подвергается сомнению: оно было получено корректными
математическими процедурами из верного утверждения (второго закона
Ньютона). Речь идёт о том, что в это уравнение не входят
термодинамические величины, в частности, поток тепла, входящий или
выходящий из системы.

Здесь становится критически важным понятие системы, которую мы
рассматриваем. Проиллюстрируем это на примере из предыдущего раздела.

Рассмотрим опять брусок, который тянут с постоянной силой $F$ по
шероховатому столу с коэффициентом трения $\mu$. Скорость бруска
постоянна. Допустим, мы протащили наш брусок на расстояние $d$. Как
определить, сколько при этом выделилось тепла? 

Уравнение \eqref{eq:c_o_m_to_pseudowork_II} ничего об этом не говорит;
как мы видели в предыдущем разделе, оно сводится к тривиальности вида
$0=0$. Попробуем применить к этой ситуации первый закон
термодинамики. Этот закон говорит нам, что внесённая в систему теплота
и совершённая над ней работа идут на изменение энергии системы:

\begin{equation}
  \label{eq:flt}
  Q^{\foot{ext}} + W^{\foot{ext}} = \Delta E. 
\end{equation}

\begin{figure}[h]
  \centering
  \begin{tikzpicture}
    \draw[interface] (5.2,0) -- (0.2,0);
    \draw[fill=gray!40,very thick] (2,0) rectangle (4,1);
    \draw[very thick,blue,->] (4,0.5) -- (5,0.5) node[above] {$F$};
    \draw[very thick,blue,->] (2,0) -- (1,0) node[above]
    {$F_{\foot{\footnotesize{тр}}}$};
    % \draw[very thick,blue,->] (3,0.5) -- (3,2) node[right] {$N$};
    % \draw[very thick,blue,->] (3,0.5) -- (3,-1) node[right] {$mg$};
    % \draw[fill=black] (3,0.5) circle (0.05);
    %
    \draw[dashed,red,thick,postaction={decorate,decoration={raise=0.1cm,text
        along path,text={|\footnotesize\color{red}|брусок + стол}}}] plot [smooth
    cycle] coordinates {(0.8,-0.2) (0.8,1.2) (4.2,1.2) (4.2,-0.2) };
  \end{tikzpicture}
  \caption{Первый закон термодинамики: брусок и стол.}
  \label{fig:flt_block_and_table}
\end{figure}

Возьмём в качестве системы брусок и стол
(см. рис. \ref{fig:flt_block_and_table}). Над этой системой совершает
работу одна внешняя сила $F$ (сила трения, которая действует со
стороны стола на брусок, является внутренней). Изменение энергии, как
мы выяснили, равно нулю, так как скорость не меняется. Таким образом,
получается, что $Q^{\foot{ext}}<0$ --- то есть, система энергию не
поглощает, а излучает наружу (как мы и ожидали) в виде теплоты. Эта
тепловая энергия, выделяющаяся в системе, равна

\begin{equation}
  \label{eq:flt_apply_I}
  Q_{\foot{брусок+стол}} = W^{\foot{ext}} = F d.
\end{equation}

Заметим, что здесь $W^{\foot{ext}}$ означает именно \textit{работу}, в
отличие от псевдоработы из предыдущего раздела. Работу силы $F$, с
которой мы тянем брусок, посчитать легко, поскольку про перемещение
точки приложения нам всё известно --- оно равно $d$. Таким образом,
вся работа, совершаемая внешней силой $F$, уходит в теплоту,
выделяющуюся в системе, то есть, за счёт этого нагреваются и стол, и
брусок.

Можно ли узнать, сколько теплоты досталось непосредственно бруску?
Здесь нас ждёт небольшой сюрприз. 

\begin{figure}[h]
  \centering
  \begin{tikzpicture}
    \draw[interface] (5.2,0) -- (0.2,0);
    \draw[fill=gray!40,very thick] (2,0) rectangle (4,1);
    \draw[very thick,blue,->] (4,0.5) -- (5,0.5) node[above] {$F$};
    \draw[very thick,blue,->] (2,0) -- (1,0) node[above]
    {$F_{\foot{тр}}$};
    % \draw[very thick,blue,->] (3,0.5) -- (3,2) node[right] {$N$};
    % \draw[very thick,blue,->] (3,0.5) -- (3,-1) node[right] {$mg$};
    % \draw[fill=black] (3,0.5) circle (0.05);
    %
    \draw[dashed,red,thick,postaction={decorate,decoration={raise=0.1cm,text
        along path,text={|\footnotesize\color{red}|брусок}}}] plot [smooth
    cycle] coordinates {(1.8,-0.2) (1.8,1.2) (4.2,1.2) (4.2,-0.2) };
  \end{tikzpicture}
  \caption{Первый закон термодинамики: брусок.}
  \label{fig:flt_block}
\end{figure}

Итак, ограничим систему для рассмотрения одним лишь бруском
(см. рис. \ref{fig:flt_block}). Для этой системы тоже можно написать
первый закон термодинамики. На этот раз внешних сил две: сила $F$, с
которой тянут брусок, и сила трения $F_{\foot{тр}}$ (равная $F$),
действующая со стороны стола. Изменение внутренней энергии системы
по-прежнему равно нулю, так как кинетическая энергия бруска не
меняется. Таким образом, выделившаяся в системе теплота равна

\begin{equation}
  \label{eq:flt_apply_II}
  Q_{\foot{брусок}} = W^{\foot{ext}} = Fd + A_{\foot{тр}}.
\end{equation}
Скорее всего, $A_{\foot{тр}}<0$, но чему она равна? Если бы она была
равна $A_{\foot{тр}} = -F_{\foot{тр}} d = - F d$, то получалось бы,
что тепло не выделяется, и брусок не нагревается, что противоречит
предыдущим выкладкам и нашему опыту. Таким образом, мы должны сделать
вывод, что работа силы трения $|A_{\foot{тр}}|$ \textit{меньше} $Fd$.

Почему так вообще может произойти? Работа получается как произведение
двух множителей: силы и перемещения точки приложения этой силы. Сила
трения в любом случае равна $F_{\foot{тр}} = F$, иначе наш брусок
ускорялся бы, вопреки условиям задачи. Значит, всё дело в перемещении
точки приложения (назовём его $x$). По каким-то причинам $x<d$, и,
таким образом, выделившаяся теплота в системе, состоящей только лишь
из бруска, равна

\begin{equation}
  \label{eq:flt_apply_III}
  Q_{\foot{брусок}} = Fd - F_{\foot{тр}}\; x = F d - F x = F(d-x).
\end{equation}

Каковы причины того, что $x<d$? Детальный ответ на этот вопрос можно
дать, рассматривая микроскопический механизм возникновения
трения\footnote{См., например, \href{http://db.tt/lXQPSHtM}{Work and
    heat transfer in the presence of sliding friction}, B.A. Sherwood
  and W.H. Bernard, \textit{Am. J. Phys}, \textbf{52} (11), November
  1984.}. В данной заметке этот вопрос не обсуждается.

Разумеется, мы получим похожий результат, если будем рассматривать
систему, состояющую только из стола. В такой системе внешней силой
является сила трения со стороны бруска. Кинетическая энергия стола,
разумеется, не меняется, и мы можем написать
\begin{equation}
  \label{eq:flt_apply_IV}
  Q_{\foot{стол}} = A_{\foot{тр}} = F_{\foot{тр}}\; x = F x. 
\end{equation}
Складывая уравнения \eqref{eq:flt_apply_IV} и \eqref{eq:flt_apply_III}
мы, естественно, получаем уравнение \eqref{eq:flt_apply_I}.

Заметим, что конкретное значение $x$ нельзя вычислить, не зная
детально параметров бруска и стола: характер контакта бруска со
столом, теплопроводность и т.д.

Мы можем применить первый закон термодинамики и ко второму примеру,
рассмотренному в разделе \ref{sec:center_of_mass_energy_work}, а
именно, к бруску, который постепенно замедляется под действием силы
трения. Если взять в качестве системы брусок + стол, то внешних сил не
будет (опять-таки, все силы трения являются в этой системе внутренними
силами), и, следовательно, теплота будет выделяться за счёт уменьшения
кинетической энергии:

\begin{equation}
  \label{eq:flt_apply_V}
  Q^{\foot{ext}} = \Delta E_{k}. 
\end{equation}
Кинетическая энергия уменьшается, так что энергия в системе
действительно выделяется. Вся кинетическая энергия уходит в нагрев
бруска и стола:
\begin{equation}
  \label{eq:flt_apply_VI}
  Q_{\foot{стол+брусок}} = \frac{mv_0^2}{2}.
\end{equation}
При этом работа силы трения, как и в предыдущем примере, не равна $FS$
(напомним, что $S$~---~путь, пройденный бруском до остановки).

\section{Как отличить работу от теплоты? }
\label{sec:diff_work_heat}

В предыдущем разделе мы использовали первый закон термодинамики в его
традиционном варианте: совершённая над системой работа плюс внесённая
теплота идут на увеличение энергии системы. Здесь уже можно задуматься
над адекватностью такого описания. Что понимается под <<внесённой
теплотой>>?

Очевидно, имеется в в виду энергия, перенесённая через границу системы
за счёт разницы температур. В более общем случае сюда же можно
включить энергию, перенесённую за счёт, например, электромагнитного
излучения (при этом эта энергия переносится опять-таки за счёт работы
электромагнитного поля над молекулами системы) или за счёт сгорания
топлива.

Последний фактор играет роль в задаче об описании движения
автомобиля. При разгоне кинетическая энергия автомобиля увеличивается,
но за счёт чего? Сила трения работы не совершает, так как точка
приложения не двигается: в системе отсчёта, связанной с дорогой, точка
соприкосновения колеса с дорогой неподвижна. Никакого подвода теплоты,
связанного с разницей температур, тоже нет. Ответ таков: кинетическая
энергия автомобиля увеличивается за счёт внутренней энергии, которая
выделяется при сгорании топлива. При этом сила трения, конечно,
ускоряет автомобиль, увеличивая скорость его центра масс.

Вернёмся к нашему равномерно двигающемуся бруску на шероховатом
столе. Очевидно, что за время движения он нагревается, но при этом он
оказывается на более холодных участках стола, следовательно, в этот
момент должна происходить теплопередача от бруска к столу. То есть,
уравнение \eqref{eq:flt_apply_III} надо записать более точно так:

\begin{eqnarray}
  \label{eq:flt_apply_q}
  \Delta E_{\foot{брусок}} & = & F d - F x -Q,\\
  \Delta E_{\foot{стол}} & = & Fx + Q.  
\end{eqnarray}

Если эти два уравнения сложить, получится, что увеличение внутренней
энергии системы ($\Delta E_{\foot{брусок}} + \Delta E_{\foot{стол}} $)
равно, разумеется, работе внешней силы $Fd$. 

Из всего этого видно, что здесь чёткой границы между работой силы
трения ($Fx$) и переданной от бруска к столу теплотой ($Q$) провести
нельзя. Тем не менее, если бы мы захотели выяснить, например, до какой
температуры нагрелись брусок и стол, мы могли бы это сделать и без
обращения к понятию теплоты и работы силы трения~---~достаточно было
бы знать полное изменение энергии системы. 

\section{Заключение.}
\label{sec:summary}

При вычислении работы надо соблюдать внимательность и не путать
\textit{работу} с \textit{псевдоработой} --- последняя служит только
для механического описания процессов и непригодна для обсуждения
тепловых потоков в системе. Вычисление работы нужно проводить вместе с
применением первого закона термодинамики, который и является законом
сохранения энергии в полном смысле.

\section{Литература.}
\label{sec:ref}

\begin{enumerate}
\item ``Work and heat transfer in the presence of sliding
    friction'', B.A. Sherwood and W.H. Bernard, \textit{Am. J. Phys},
  \textbf{52} (11), November 1984.
\item ``Physics that Textbook Writers Usually Get Wrong'', Robert
  P. Bauman, \textit{Phys. Teach.}, \textbf{30}, 353 (1992).
\item ``Energy and the Confused Student'', John W. Jewett,
  \textit{Phys. Teach.}, \textbf{46}, 38 (2008).
\end{enumerate}

\end{document}


%%% Local Variables: 
%%% mode: latex
%%% TeX-engine:xetex
%%% TeX-PDF-mode: t
%%% End: 

%%% Local Variables: 
%%% mode: latex
%%% TeX-engine:xetex
%%% TeX-PDF-mode: t
%%% End:
