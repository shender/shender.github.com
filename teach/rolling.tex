\documentclass[a4paper,12pt]{article}
% math symbols
\usepackage{amssymb,amsmath}
\synctex=1
% for different compilers
\usepackage{ifpdf}
% geometry of page
\usepackage[margin=2cm]{geometry}

% if pdflatex, then
\ifpdf
\usepackage[russian]{babel}
\usepackage[utf8]{inputenc}
\usepackage[unicode]{hyperref}
\usepackage[pdftex]{graphicx}
\usepackage{cmlgc}
% if xelatex, then
\else
% math fonts
\usepackage{fouriernc}
% xelatex specific packages
\usepackage[xetex]{hyperref}
\usepackage{xltxtra}	% \XeLaTeX macro
\usepackage{xunicode}	% some extra unicode support
\defaultfontfeatures{Mapping=tex-text}
\usepackage{polyglossia}	% instead of babel in xelatex
\usepackage{indentfirst}	% 
\setdefaultlanguage{russian}
% fonts
\newfontfamily\cyrillicfont{SchoolBookC}
\newfontfamily\cyrillicfontsf{TextBookC}
\setmonofont{Consolas}
\fi

% several pictures in one figure
\usepackage{subfig}
% calc in TeX expressions
\usepackage{calc}
% nice pictures and plots
\usepackage{pgfplots,tikz,circuitikz}
% different libraries for pictures
\usetikzlibrary{%
  arrows,%
  calc,%
  patterns,%
  decorations.pathreplacing,%
  decorations.pathmorphing,%
  decorations.markings,%
  intersections,%
  decorations.text%
}

\usepackage{tkz-euclide}

\graphicspath{{./pics/}}

% colors of the hyperlinks
\hypersetup{colorlinks,%
  citecolor=blue,%
  urlcolor=blue,%
  linkcolor=red
}

\newcommand{\foot}[1]{\mbox{\footnotesize{#1}}}

\tikzset{>=latex,%
  interface/.style={postaction={draw,decorate,decoration={border,angle=45,amplitude=0.2cm,segment
        length=1mm}}},%
  spring/.style={thick,decorate,decoration={aspect=0.5, segment
  length=1.3mm, amplitude=1mm,coil}},%
  marrow/.style={postaction={draw,decorate,decoration={markings,
        mark=at position 0.6 with {\arrow{latex}}}}}}

% обозначение угла
\tikzset{arcnode/.style={
            decoration={
                        markings, raise = 3mm,  
                        mark=at position 0.5 with { 
                                    \node[inner sep=0] {#1};
                        }
            },
            postaction={decorate}
      }
}

% команда для отметки угла: проводит дугу между двумя лучами,
% проведёнными между точками #2--#3 и #2--#4
% первый аргумент - необязательный, стиль линии
% #2,#3,#4 - точки
% #5 - радиус дуги для обозначения угла
% #6 - обозначение угла
\newcommand*\marktheangle[6][]{
            \draw[thick,arcnode={#6},#1] let \p2=($(#3)-(#2)$),%
                        \p3=($(#4)-(#2)$),%
                        \n2 = {atan2(\x2,\y2)},%
                        \n3 = {atan2(\x3,\y3)}%
                        in ($(\n2:#5)+(#2)$) arc (\n2:\n3:#5);
            }


\pgfplotsset{compat=newest}

\begin{document}
%\parindent=0mm

\begin{center}
  \LARGE{Некоторые аспекты физики вращения}\\[0.2cm]
  \large{Игорь Шендерович}\\[0.2cm]
  \normalsize{\texttt{shender.i@gmail.com}}
\end{center}

\begin{abstract}
  Выводится уравнение вращательного движения в общем виде,
  относительно любой точки (не только мгновенного центра
  вращения). Метод иллюстрируется различными примерами. 
\end{abstract}

\section{Общее уравнение вращательного движения. }
\label{sec:general_eom}

\begin{figure}[h]
  \centering
  \begin{tikzpicture}
    \def\r{1.5cm};    
    \coordinate (a) at (10,0); 
    \coordinate (O) at (3cm,\r);
    \coordinate (C) at ($(O) + (110:\r/2)$);
    \coordinate (P) at (8,0);
    \coordinate (i) at ($(C) + (-80:\r/1.5)$);
    \draw[interface,thick] (a) -- (1,0);    
    \draw[fill=gray!20] (O) circle (\r);
    \draw[fill=black] (C) circle (0.05cm) node[above,blue] {$C$};
    \draw[fill=black] (i) circle (0.05cm) node[below,blue] {$i$};
    
    \draw[very thick,blue,->] (P) node[above] {$P$} -- (C)
    node[pos=0.5,above] {$\vec{r}_{C,P}$}; 
    \draw[very thick,blue,->] (P)  -- (i)
    node[pos=0.65,below] {$\vec{r}_{i,P}$};
    \draw[very thick,blue,->] (C) -- (i) node[midway,left]
    {$\vec{r}_{i,C}$};
  \end{tikzpicture}
  \caption{Качение тела по горизонтальной поверхности.}
  \label{fig:rolling_general}
\end{figure}

Пусть $C$ --- центр масс катящегося тела, а $P$ --- произвольная точка
в нашей системе. Обратим внимание, что $P$ может быть не привязана к
физическим телам. Момент импульса тела относительно оси, проходящей
через $P$, даётся выражением

\begin{equation}
  \label{eq:momentum_p_general_1}
  \vec{L}_{P} = \sum_i m_i \vec{r}_{i,P} \times \vec{v}_{i,P} = \sum_i
  m_i \left( \vec{r}_{i,C} + \vec{r}_{C,P} \right) \times \left(
    \vec{v}_{i,C} + \vec{v}_{C,P} \right). 
\end{equation}
Раскроем скобки в этом выражении. Одно слагаемое из четырёх не будет
зависеть от точки $P$, назовём его <<спиновой>> составляющей момента
импульса:

\begin{equation}
  \label{eq:L_spin_def}
  \vec{L}^s \equiv \sum_i m_i \vec{r}_{i,C} \times \vec{v}_{i,C}. 
\end{equation}
В случае двумерного движения, как известно, $\vec{L}^s = I_c
\vec{\omega}$, где $I_C$~---~момент инерции тела относительно оси,
проходящей через центр масс.

Ещё одно слагаемое не будет зависеть от рассматриваемой точки
$i$. Назовём его <<орбитальной>> составляющей момента импульса:

\begin{equation}
  \label{eq:L_orb_def}
  \vec{L}_P^o \equiv \sum_i m_i \vec{r}_{C,P} \times \vec{v}_{C,P} = M
  \vec{r}_{C,P} \times \vec{v}_{C,P}.
\end{equation}

Что с остальными двумя слагаемыми? Посмотрим на одно из них.

\begin{equation}
  \label{eq:L_trash_1}
  \sum_i m_i \vec{r}_{i,C} \times \vec{v}_{C,P} = \left( \sum_i m_i
    \vec{r}_{i,C} \right) \times \vec{v}_{C,P}. 
\end{equation}
То, что стоит в скобках --- радиус--вектор центра масс, сосчитанный в
системе отсчёта центра масс, то есть, ноль. Таким образом, всё
слагаемое равно нулю. То же самое можем заключить относительно второго
слагаемого.

Таким образом, полный момент импульса системы относительно точки $P$
складывается из <<орбитальной>> и <<спиновой>> частей:

\begin{equation}
  \label{eq:L_full_general}
  \vec{L}_{P} = \vec{L}_P^o + \vec{L}^s = M \vec{r}_{C,P} \times
  \vec{v}_{C,P} + I_C \vec{\omega}. 
\end{equation}

Вычислим теперь производную $L_{P}$ по времени, исходя из определения
\eqref{eq:momentum_p_general_1}:
\begin{equation}
  \label{eq:dL/dt}
  \frac{d\vec{L}_P}{dt} = \sum_i m_i \vec{v}_{i,P} \times
  \vec{v}_{i,P} + \sum_i m_i \vec{r}_{i,P} \times \vec{a}_{i,P}. 
\end{equation}
Первое слагаемое, очевидно, равно нулю, а второе представляет собой
векторное произведение из радиус--вектора, проведённого в точку $i$ из
точки $P$ на силу, действующую на точку $i$, то есть, момент этой силы
относительно точки $P$.

Таким образом, уравнение движения для вращающегося тела может быть
записано в следующей форме:

\begin{equation}
  \label{eq:eom_general}
  \vec{\tau}_P = \sum_i \vec{r}_{i,P} \times \vec{F}_{i,P} =
  \frac{d}{dt} \left(M \vec{r}_{C,P} \times \vec{v}_{C,P} + I_C
    \vec{\omega}  \right) =   \frac{d\vec{L}_P}{dt}. 
\end{equation}
Это и есть наиболее общее и корректное уравнение моментов относительно
произвольной точки $P$. Оно позволяет решать многие задачи не прибегая
к дополнительным соображениям (мы даже не обсуждали мгновенную ось
вращения при выводе этих формул). Рассмотрим несколько примеров.

\section{Шар для боулинга: скольжение и качение. }
\label{sec:bowling_sliding_rolling}

Рассмотрим однородный шар радиуса $R$, запущенный без вращения со
скоростью $V_0$ вдоль горизонтальной поверхности. Вычислим скорость,
при которой проскальзываение сменится качением. Отсутствие
проскальзывания означает, что скорость точки касания шара с полом
равна нулю.

Применим формулу \eqref{eq:eom_general}. В качестве точки $P$ возьмём
сначала любую точку на расстоянии $R$ от пола. Моменты сил $N$ и $Mg$
относительно точки $P$ взаимно сокращаются, и остаётся только момент
силы трения, равный $\tau_P = \mu M g R$. Теперь надо найти момент
импульса тела относительно точки $P$.

Сделать это просто: очевидно, орбитальная часть $L_P$ обратится в
ноль, так что остаётся только спиновая часть. Таким образом, получим,
что

\begin{equation}
  \label{eq:ball_eom_1}
  \mu M g R = \frac{2}{5} MR^2 \dot{\omega}. 
\end{equation}
Мы учли, что момент инерции шарика относительно центра масс равен $I_C
= 2/5 MR^2$. 

Это почти тривиальное утверждение, говорящее, что именно сила трения
раскручиывает наш шарик (изменяет угловую скорость, создавая
постоянное угловое ускорение).

Теперь возьмём в качестве точки $P$ какую-нибудь точку на поверхности
пола. Моменты силы тяжести и силы реакции опять скомпенсированы, но на
этот раз обнуляется и момент силы трения. Таким образом, суммарный
момент сил относительно точки $P$ равен нулю, а это означает, что
момент импульса относительно этой точки сохраняется.

В самом начале движения, когда шарик ещё не крутится, этот момент
импульса равен $L_0 = R M V_0$. В произвольный момент времени, когда у
шарика есть угловая скорость $\omega$ и скорость центра масс $V$, этот
момент импульса равен (по формуле \eqref{eq:L_full_general})  $L_1 =
RMV + I_C \omega $. Из сохранения момента импульса следует, что

\begin{equation}
  \label{eq:conservation_L_ball}
  RMV + I_C \omega = RMV_0. 
\end{equation}

В тот момент, когда заканчивается проскальзываение, скорость точек на
поверхности шара становится равной $V = \omega R$. Отсюда получаем,
что

\begin{equation}
  \label{eq:conservation_L_ball_V}
  V = \frac{V_0}{1+ I_C/MR^2} = \frac{5}{7} V_0. 
\end{equation}

\section{Цилиндр и наклонная плоскость.}
\label{sec:cylinder_incline}

Усложним немного задачу --- сделаем плоскость наклонной. Итак, пусть
имеется наклонная шероховатая поверхность с углом $\alpha$ в
основании. На поверхность ставится полый цилиндр массы $M$ и радиуса
$R$, вращающийся с угловой скоростью $\omega_0$. Через какое время
цилиндр достигнет наивысшего положения на наклонной плоскости?

\begin{figure}[h]
  \centering
  \begin{tikzpicture}
    \def\r{0.7cm};
    \def\angle{20};
    \def\size{5};
    \coordinate (a) at (0,0);
    \coordinate (b) at ($({\size*cos(\angle)},0)$);
    \coordinate (c) at ($({\size*cos(\angle)},{\size*sin(\angle)})$);

    \draw[thick] (c) -- (a) -- (b);
    \node[draw,very thick,rotate around={\angle:(a)},circle,inner
    sep=0cm, minimum size=2*\r] (ball) at (\size/2,\r) {};
    \draw[very thick,red,->] (ball.center) node[above,blue] {$C$} --
    ++(0,-1.3cm); 

    \coordinate (P) at ($(a)!(ball.center)!(c)$);
    \draw[very thick,red,->] (P) node[above,blue] {$P$} --
    ++(\angle:1.3cm) node[above] {$F_{\mbox{\footnotesize{тр}}}$};
    

    \marktheangle[blue]{a}{c}{b}{1}{$\alpha$};
  \end{tikzpicture}
  \caption{Цилиндр, вкатывающийся на наклонную плоскость.}
  \label{fig:cylinder_incline}
\end{figure}

Здесь опять можно применить уравнение \eqref{eq:eom_general}. Запишем
его относительно точки $P$, показанной на рисунке.

Момент относительно этой точки создаётся силой тяжести. Её плечо
равно, очевидно, $R \sin \alpha$, и момент, таким образом, равен $\tau
= -MgR \sin \alpha$.

Момент импульса в каждый момент времени имеет и орбитальную, и
спиновую часть; благодаря симметричности ситуации, запишем

\begin{equation}
  \label{eq:L_cylinder_1}
  L (t) = RM v(t) + I_C \omega (t). 
\end{equation}

Уравнение движения \eqref{eq:eom_general} даёт

\begin{equation}
  \label{eq:L_cylinder_2}
  R M d v(t) + I_C d \omega(t) = -MgR \sin \alpha \cdot dt. 
\end{equation}

Проинтегрируем это уравнение от начального момента $t=0$ до момента
достижения высшей точки $t=T$ (напомним, $T$ --- время, которое надо
найти). 

\begin{equation}
  \label{eq:L_cylinder_3}
  RM [ v(T) - v(0)] + I_C [ \omega(T) - \omega(0) ] = - MgR \sin
  \alpha \cdot T. 
\end{equation}
По условию задачи $v(0)=0$ (изначально поступательной скорости у
цилиндра не было), $\omega(0) = \omega_0$. В наивысшей точке,
очевидно, $v(T)=\omega(T)=0$, так что получаем
\begin{equation}
  \label{eq:L_cylinder_answer}
  T = \frac{I_C\omega_0}{MgR \sin \alpha} = \frac{\omega_0 R}{g \sin
    \alpha }. 
\end{equation}

\section{Качение полукольца. }
\label{sec:semihoop}

Применим теперь общий подход к задаче о качении полукольца. Пусть
в вертикальной плоскости имеется полукольцо массы $M$ и радиуса
$R$. Кольцо может качаться (без проскальзывания) по горизонтальной
поверхности. Как будет выглядеть уравнение движения этого полукольца?

Попробуем получить уравнение зависимости угла $\theta$ (см. рис.) от
времени. 

\begin{figure}[h]
  \centering
  \begin{tikzpicture}
    \def\r{3cm};
    \def\angle{30};
    % горизонталь
    \coordinate (a) at (-2,0);
    % центр полуокружности
    \coordinate (O) at (5cm-\r,\r);
    % точка касания с полом
    \coordinate (P) at ( O |- a);
    % центр масс
    \coordinate (C) at ($(O) + (\angle-90:2*\r/3.14)$);
    \draw[fill=black] (C) circle (0.07cm); 
    % точка B
    \coordinate (B) at ($(O)!(C)!(P)$);
    % пол
    \draw[interface,thick] (6,0) -- (a);
    % полуокружность
    \draw[line width=0.7mm,rotate around={\angle:(O)}] (5cm,\r) arc
    (0:-180:\r);
    % геометрия
    \draw[thin,blue] (O) node[above] {$O$}  -- (C) node[right]
    {$C$} -- (P) node[below=0.2cm] {$P$} -- (O) node[pos=0.7,left]
    {$R$};
    % сила тяжести
    \draw[red,very thick,->] (C) -- ++(0,-2cm) node[right] {$Mg$};
    % пунктиры
    \draw[dashed] (C) -- (B) node[left,blue] {$B$};
    \draw[dashed,rotate around={\angle:(O)}] (5cm,\r) --
    ++(-2*\r,0);
    \marktheangle[blue]{O}{C}{P}{0.6cm}{$\theta$};
   \end{tikzpicture}
  \caption{Качающееся полукольцо.}
  \label{fig:semihoop}
\end{figure}

Будем рассматривать движение вокруг точки $P$ --- точки касания
полукольца с горизонтальной поверхностью. К полукольцу приложены три
силы: сила тяжести, сила реакции опоры и сила трения, но момент вокруг
точки $P$ создаёт лишь первая из них (остальные две силы приложены в
точке $P$ и не отмечены для облегчения рисунка). 

Плечо этой силы равно длине отрезка $BC$. Найти её можно из
прямоугольного треугольника $\triangle OBC$. Известно, что для
однородного полукольца радиуса $R$ расстояние от геометрического
центра $O$ до центра масс $C$ равно

\begin{equation}
  \label{eq:OC}
  OC = \frac{2R}{\pi},
\end{equation}
так что момент силы $Mg$ вокруг точки $P$ равен

\begin{equation}
  \label{eq:torque_mg}
  \tau_P = -Mg \cdot BC = -Mg \cdot OC \cdot \sin \theta = -\frac{2MgR \sin
    \theta}{\pi}.  
\end{equation}

Для того, чтобы написать уравнение движения, надо найти момент
импульса относительно точки $P$ согласно общей формуле
\eqref{eq:eom_general}. Он состоит из <<орбитальной>> и <<спиновой>>
частей. <<Орбитальная>> часть равна

\begin{equation}
  \label{eq:l_orb_semihoop_1}
  \vec{L}_{P}^{o} = \vec{r_{C,P}} \times M \vec{v}_{C,P}. 
\end{equation}

Будем честно вычислять это векторное произведение. Радиус-вектор точки
$C$ относительно точки $P$ состоит имеет две компоненты: $BC$ и
$PB$. Как видно из рисунка,
\begin{equation}
  \label{eq:r_c_p}
  BC = x(\theta) = \frac{2R\sin \theta}{\pi}, \quad PB
  = y (\theta) = R \left( 1 - \frac{2\cos \theta}{\pi} \right).
\end{equation}

Отсюда же можно найти скорость центра масс относительно точки
$P$. Дифференцируя соотношения \eqref{eq:r_c_p} и не забывая, что
$\theta = \theta (t)$, получим

\begin{equation}
  \label{eq:v_c_p}
  \dot {x} = \frac{2R \cos \theta \cdot \dot{\theta} }{\pi}, \quad
  \dot{y} = \frac{2R \sin \theta \cdot \dot{\theta} }{\pi}.
\end{equation}

Итак, мы нашли компоненты векторов в векторном произведении
\eqref{eq:l_orb_semihoop_1}. Отсюда легко найти сам вектор
орбитального момента импульса. Он будет направлен перпендикулярно
плоскости рисунка, а его длина равна

\begin{equation}
  \label{eq:l_orb_semihoop_2}
  L_P^o = M \left( x(\theta) \dot{y} (\theta) - y(\theta) \dot{x}
    (\theta)  \right)  = 
  \frac{4MR^2 \dot{\theta}}{\pi^2} - \frac{2MR^2 \cos \theta
    \cdot \dot{\theta}}{\pi}. 
\end{equation}

Спиновая часть вычисляется совсем просто. Она, как мы видели, не
зависит от точки $P$ и равна 

\begin{equation}
  \label{eq:l_spin_semihoop_1}
  L^s = I_C \dot{\theta} = \left( I_O - M \cdot OC^2 \right) \dot
  {\theta} = \left( MR^2 - \frac{4MR^2}{\pi^2} \right) \dot{\theta} =
  MR^2 \dot{\theta} - \frac{4MR^2 \dot{\theta}}{\pi^2}. 
\end{equation}
Мы использовали здесь теорему Штейнера, а также тот элементарный факт,
что момент инерции полукольца относительно точки $O$ равен $MR^2$. 

Итак, момент импульса полукольца относительно точки $P$ даётся
выражением
\begin{equation}
  \label{eq:l_full_semihoop}
  L_P = L_P^o + L^s = MR^2 \dot{\theta} - \frac{2MR^2 \cos \theta
    \cdot \dot{\theta}}{\pi}.
\end{equation}

Производная $L_P$ по времени должна быть равна моменту силы тяжести
$\tau_{P}$. Итак, искомое уравнение движения записывается в виде
\begin{equation}
  \label{eq:eom_semihoop}
  \ddot{\theta} \left( 1- \frac{2 \cos \theta}{\pi} \right) + 2
  \dot{\theta}^2 \frac{\sin \theta}{\pi} = - \frac{2g \sin \theta}{\pi
  R}. 
\end{equation}

Заметим, что при $\theta \ll 1$ получаются обыкновенные гармонические
колебания: вторым слагаемым в левой части можно пренебречь, скобка в
первом слагаемом даёт константу, а $\sin \theta$ в правой части
превращается в $\theta$. 

Разумеется, это уравнение движения можно было получить и из других
соображений --- например, из закона сохранения энергии. У полукольца
есть кинетическая энергия вращательного движения и потенциальная
энергия, связанная с тем, что центр масс меняет свою высоту. При этом
полная энергия сохраняется, так как сила трения не совершает работу
(перемещение её точки приложения равно нулю). Вращение происходит
вокруг мгновенной оси вращения, которая проходит через точку $P$.
Запишем
\begin{equation}
  \label{eq:conserve_energy_semihoop}
  E = \frac{I_P \dot{\theta}^2}{2} + Mg \frac{2R}{\pi} \left(
    1- \cos \theta \right). 
\end{equation}

Момент инерции $I_P$ относительно точки $P$ опять проще всего получить по
теореме Штейнера:

\begin{equation}
  \label{eq:steiner_semihoop}
  I_P (\theta) = I_C + M \cdot CP^2 = I_C + M \left(
    \frac{4R^2}{\pi^2} + R^2 - \frac{4R^2}{\pi} \cos \theta \right) =
  2 MR^2 - \frac{4MR^2}{\pi} \cos \theta.
\end{equation}
Мы использовали здесь теорему косинусов для $\triangle OCP$. Заметим,
что момент инерции зависит от угла $\theta$, который, в свою очередь,
зависит от времени $t$. Это значит, что дифференцируя уравнение
\eqref{eq:conserve_energy_semihoop} по времени, нам придётся также
учитывать производную по времени от $I_P$.

Подставляя \eqref{eq:steiner_semihoop} в
\eqref{eq:conserve_energy_semihoop} и дифференцируя по времени,
получим уравнение движения \eqref{eq:eom_semihoop}. 

Частая ошибка решения таких задач состоит в том, что упускается из
виду <<орбитальная>> часть момента импульса, а вместо этого пишется
$\tau = I_P \ddot{\theta}$. Разумеется, получающиеся таким образом уравнения
движения неверны, хотя при малых $\theta$ дают такое же приближённое
решение. 

\end{document}


%%% Local Variables: 
%%% mode: latex
%%% TeX-engine:xetex
%%% TeX-PDF-mode: t
%%% End: 