\documentclass[11pt]{article}
\synctex=1
% math symbols
\usepackage{amssymb,amsmath}
% for different compilers
\usepackage{ifpdf}
% geometry of page
\usepackage[margin=2cm]{geometry}
% float pictures
\usepackage{wrapfig}
\usepackage{enumitem}

% if pdflatex, then
\ifpdf
 \usepackage[english,russian]{babel}
 \usepackage[utf8]{inputenc}
 \usepackage[unicode]{hyperref}
 \usepackage[pdftex]{graphicx}
 \usepackage{cmlgc}
% if xelatex, then
\else
% math fonts
 \usepackage{fouriernc}
% xelatex specific packages
 \usepackage[xetex]{hyperref}
 \usepackage{xunicode}	% some extra unicode support
 \usepackage{xltxtra}	% \XeLaTeX macro
 \defaultfontfeatures{Mapping=tex-text}
 \usepackage{polyglossia}	% instead of babel in xelatex
 \setdefaultlanguage{russian}
% fonts
\newfontfamily\cyrillicfont{JournalC}
% \newfontfamily\cyrillicfont{Old Standard}
% \newfontfamily\cyrillicfont{SchoolBookC}
\newfontfamily\cyrillicfontsf{TextBookC}
\fi


\tolerance=1000
\emergencystretch=0.74cm
\pagestyle{empty}

\begin{document}

\begin{center}
  \Large{\textbf{Физический хоккей.}}
\end{center}



\begin{enumerate}[label*=\arabic*.]
  \setlength{\itemsep}{-1mm}
  \item \textbf{Ход матча}. 
  \begin{enumerate}[label*=\arabic*.]
  \item В физическом хоккее участвуют две команды.
  \item Каждый игрок является вратарём, защитником или нападающим.
  \item Игровое поле состоит из 5 зон: зона ворот 1-й команды, зона
    защиты 1-й команды, центральная зона, зона защиты 2-й команды,
    зона ворот 2-й команды. Игровое поле изображается на доске либо
    представляется в виде макета.
  \item Шайба в любой момент находится в одной из зон. Шайба
    изображается на доске либо размещается на макете.
  \item В начале матча шайба располагается в центральной зоне.
  \item Матч состоит из розыгрыша некоторого количества задач. Матч
    заканчивается, когда разыграно заранее определенное число задач
    (например, 15) либо истекло отведённое время (например, 45 минут).
  \item Розыгрыш задачи может закончиться победой одной из команд,
    либо ничьей.
  \item Если команда побеждает в розыгрыше задачи при том, что шайба
    находится в зоне ворот противоположной команды, то победившей
    команде засчитывается гол, и шайба перемещается в центральную
    зону.
  \item Если команда побеждает в розыгрыше задачи и шайба не находится
    в зоне ворот противоположной команды, то шайба приближается на
    одну зону к зоне ворот противоположной команды.
  \item В случае, если розыгрыш задачи завершился ничьей, шайба
    остается в той же зоне.
  \item В матче побеждает команда, забившая больше голов.
  \end{enumerate}
  \item \textbf{Розыгрыш задачи}.
    \begin{enumerate}[label*=\arabic*.]
    \item Если шайба находится в центральной зоне, то в розыгрыше
      участвуют нападающие и защитники обеих команд.
    \item Если шайба находится в зоне защиты одной из команд, то в
      розыгрыше участвуют защитники этой команды и нападающие другой
      команды.
    \item Если шайба находится в зоне ворот одной из команд, то в
      розыгрыше участвуют вратарь этой команды и нападающие другой
      команды.
    \item Игрокам зачитывается условие задачи. В течение основного
      времени (например, 4 минуты) любой из участвующих в розыгрыше
      игроков может дать ответ.
    \item Если до истечения основного времени розыгрыша ни один игрок
      не дал ответа на задачу, то все оставшиеся игроки команд
      присоединяются к розыгрышу и получают возможность давать ответ в
      течение дополнительного времени (например, 2 минуты) ---
      т.н. \textit{набегание}.
    \item Если игрок дает верный ответ, то его команда победила в
      розыгрыше.
    \item Если игрок дает неверный ответ, то его команда проиграла в
      розыгрыше.
    \item Если ни один игрок не дал ответа на задачу в течение
      основного и дополнительного времени, то розыгрыш закончился
      вничью.
    \end{enumerate}
    \clearpage
  \item \textbf{Рекомендации}.
    \begin{enumerate}[label*=\arabic*.]
    \item Оптимальный состав команды --- 1 вратарь, 2 защитника, 2
      нападающих (всего 5 человек). Если нужно сделать команду
      большего или меньшего размера, можно изменить число защитников
      или нападающих.
    \item Задачи должны быть достаточно простыми, чтобы большая их
      часть решалась за отведенные 4--6 минут.
    \item Игрокам следует дать возможность сменить роли в середине
      матча либо между матчами.
    \item Желательно иметь два стола (или один большой стол) для
      игроков, участвующих в розыгрыше задачи. Игроки занимают места
      за ними в соответствии с положением шайбы либо с началом
      дополнительного времени.
    \item Данный игроком числовой ответ, отличающийся от ожидаемого не
      более чем на 10\%, считается верным.
    \item При большом числе команд (6 и более) рекомендуется
      проводить матчи в рамках турнира --- например, по
      \href{http://ru.wikipedia.org/wiki/%D0%A8%D0%B2%D0%B5%D0%B9%D1%86%D0%B0%D1%80%D1%81%D0%BA%D0%B0%D1%8F_%D1%81%D0%B8%D1%81%D1%82%D0%B5%D0%BC%D0%B0}{швейцарской системе.}
    \end{enumerate}
\end{enumerate}


\end{document}
