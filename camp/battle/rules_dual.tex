\documentclass[11pt]{article}
\synctex=1
% math symbols
\usepackage{amssymb,amsmath}
% for different compilers
\usepackage{ifpdf}
% geometry of page
\usepackage[margin=2cm]{geometry}
% float pictures
\usepackage{wrapfig}

% if pdflatex, then
\ifpdf
 \usepackage[english,russian]{babel}
 \usepackage[utf8]{inputenc}
 \usepackage[unicode]{hyperref}
 \usepackage[pdftex]{graphicx}
 \usepackage{cmlgc}
% if xelatex, then
\else
% math fonts
 \usepackage{fouriernc}
% xelatex specific packages
 \usepackage[xetex]{hyperref}
 \usepackage{xunicode}	% some extra unicode support
 \usepackage{xltxtra}	% \XeLaTeX macro
 \defaultfontfeatures{Mapping=tex-text}
 \usepackage{polyglossia}	% instead of babel in xelatex
 \setdefaultlanguage{russian}
% fonts
 \setromanfont{Charis SIL}
 \setsansfont{OfficinaSansC} 
 \setmonofont{Consolas}
\fi


\tolerance=1000
\emergencystretch=0.74cm
\pagestyle{empty}

\begin{document}

\begin{center}
  \Large{\textbf{Двойной физбой. Правила.}}
\end{center}

\begin{enumerate}
\setlength{\itemsep}{-1mm}
\item В физбое участвуют две команды.
\item Командам выдается определенное количество задач. Действует
  «джентльменское правило», согласно которому тот, кто знает
  какую-либо из выдаваемых задач вместе с решением, должен сразу
  заявить об этом. После этого командам отводится время на решение
  задач.
\item В каждой команде есть капитан и заместитель капитана. Все
  общение команд с жюри производится через капитанов (или через
  заместителя капитана, если капитан находится у доски).
\item После решения задач проводится конкурс капитанов. Капитанам
  сообщается одна задача. Тот из капитанов, кто первым решит задачу,
  сообщает свое решение жюри. Если оно верно, этот капитан
  выиграл. Если нет, то конкурс выиграл другой капитан.
\item Победивший капитан выбирает, какая из команд делает первый
  вызов.
\item Команда, обладающая правом вызова, может вызвать другую команду
  на любую из еще не разобранных задач.
\item Вызванная команда может принять вызов или проверить корректность вызова.
\item Если вызов принят, вызванная команда выставляет отвечающего, а
  вызывавшая --- оппонента. В случае проверки корректности вызванная
  команда выставляет оппонента, а вызывавшая --- отвечающего.
\item Отвечающий и оппонент имеют право взять с собой к доске и
  использовать любые предметы. Но, выйдя к доске, они не имеют права
  возвращаться за какими-либо предметами, а также общаться с командой.
\item Отвечающий рассказывает решение задачи оппоненту. Оппонент имеет
  право задавать вопросы и высказывать замечания по решению, а
  отвечающий выбирает, будет ли это происходить по ходу решения или по
  его окончании.
\item Если вызов был принят, а оппонент продемонстрировал неверность
  решения отвечающего, оппонент имеет право рассказать свое
  решение. При этом отвечающий имеет право задавать вопросы по решению
  оппонента.
\item После окончания решения и вопросов жюри задает свои вопросы по
  решению отвечающего или оппонента.
\item Если при проверке корректности оппонент показал неверность
  решения отвечающего, вызов признается некорректным.
\item В случае некорректного вызова вызывавшая команда получает –6
  очков, вызванная команда 0 очков. В остальных случаях: отвечающий за
  свое решение получает от 0 до 12 очков, оппонент за свое решение
  получает от 0 до 6 очков, отвечающий и оппонент за вопросы и
  замечания получают от 0 до 6 очков (6 очков даются за демонстрацию
  неверности решения). Общее количество очков, набранное командами и
  жюри, в любом случае равно 12.
\item В случае некорректного вызова право вызова на следующую задачу
  остается у той же команды. В любом другом случае оно переходит к
  другой команде.
\item Команда, обладающая правом вызова, имеет право отказаться от
  ведения боя. При этом другая команда получает право рассказать
  решения всех еще не разобранных задач. Отказавшаяся от ведения боя
  команда имеет право выставлять оппонентов, но не получает очков за
  оппонирование.
\item Каждый участник боя имеет право выходить к доске в качестве
  отвечающего или оппонента не более заранее определенного количества
  раз.
\item Если физбой заканчивается вничью, но необходимо определить
  победителя, командам выдается одна дополнительная
  задача. Определение победителя проходит по правилам конкурса
  капитанов, с той разницей, что задачу может решать вся команда.
\end{enumerate}


\end{document}
