\documentclass[11pt]{article}
\synctex=1
% math symbols
\usepackage{amssymb,amsmath}
% for different compilers
\usepackage{ifpdf}
% geometry of page
\usepackage[top=0.6cm,left=1.25cm,right=1.25cm,bottom=1.25cm]{geometry}
% float pictures
\usepackage{wrapfig}

% if pdflatex, then
\ifpdf
 \usepackage[english,russian]{babel}
 \usepackage[utf8]{inputenc}
 \usepackage[unicode]{hyperref}
 \usepackage[pdftex]{graphicx}
 \usepackage{cmlgc}
% if xelatex, then
\else
% math fonts
 \usepackage{fouriernc}
% xelatex specific packages
 \usepackage[xetex]{hyperref}
 \usepackage{xunicode}	% some extra unicode support
 \usepackage{xltxtra}	% \XeLaTeX macro
 \defaultfontfeatures{Mapping=tex-text}
 \usepackage{polyglossia}	% instead of babel in xelatex
 \setdefaultlanguage{russian}
% fonts
 \setromanfont{Charis SIL}
 \setsansfont{OfficinaSansC} 
 \setmonofont{Consolas}
\fi


\tolerance=1000
\emergencystretch=0.74cm
\pagestyle{empty}

\begin{document}

\begin{center}
  \Large{\textbf{Тройной физбой. Правила.}}
\end{center}

\begin{enumerate}
\setlength{\itemsep}{-2mm}
\item В физбое участвуют три команды.
\item Командам выдается определенное количество задач. Действует
  «джентльменское правило», согласно которому тот, кто знает
  какую-либо из выдаваемых задач вместе с решением, должен сразу
  заявить об этом. После этого командам отводится время на решение
  задач.
\item В каждой команде есть капитан и заместитель капитана. Все
  общение команд с жюри производится через капитанов (или через
  заместителя капитана, если капитан находится у доски).
\item После решения задач проводится конкурс капитанов. Капитанам
  сообщается одна задача. Победителем является капитан, первым
  рассказавший верное решение задачи. Если два капитана рассказали
  неверное решение, победителем является третий капитан. Каждый
  капитан имеет право рассказывать только одно решение.
\item Вызовы производятся командами в следующем порядке: $1
  \rightarrow 2$, $2 \rightarrow3$, $3 \rightarrow 1$, $1
  \rightarrow 3$, $3 \rightarrow 2$, $2 \rightarrow 1$. Этот
  порядок может быть нарушен, только если одна из команд откажется от
  ведения боя. Победитель конкурса капитанов определяет, какая команда
  делает первый вызов и какая команда его принимает.
\item Команда, обладающая правом вызова, может вызвать другую команду
  на любую из еще не разобранных задач.
\item Вызванная команда может принять вызов или проверить корректность
  вызова.
\item Если вызов принят, то вызывавшая команда выставляет оппонента,
  вызванная --- отвечающего, а третья команда --- рецензента. В случае
  проверки корректности вызывавшая команда выставляет отвечающего,
  вызванная --- оппонента, а третья команда --- рецензента.
\item Отвечающий, оппонент и рецензент имеют право взять с собой к
  доске и использовать любые предметы. Но, выйдя к доске, они не имеют
  права возвращаться за какими-либо предметами, а также общаться с
  командой.
\item Отвечающий рассказывает решение задачи. Оппонент имеет право на
  вопросы по решению, а отвечающий выбирает, будут ли они заданы в
  ходе решения или по его окончании. Рецензент имеет право на вопросы
  по окончании вопросов оппонента.
\item Если вызов был принят, а оппонент показал неверность решения
  отвечающего, то оппонент получает право рассказать свое решение
  задачи. При этом рецензент имеет преимущественное право на
  вопросы. Если отвечающий или рецензент покажут неверность решения
  оппонента, то рецензент получает право рассказать свое решение
  задачи. При этом отвечающий имеет преимущественное право на вопросы.
\item Если вызов был принят, а рецензент показал неверность решения
  отвечающего, то рецензент получает право рассказать свое решение
  задачи. При этом оппонент имеет преимущественное право на
  вопросы. Если отвечающий или оппонент покажут неверность решения
  рецензента, то оппонент получает право рассказать свое решение
  задачи. При этом отвечающий имеет преимущественное право на вопросы.
\item Если при проверке корректности оппонент или рецензент покажут
  неверность решения отвечающего, то рецензент получает право
  рассказать свое решение. При этом оппонент имеет преимущественное
  право на вопросы, а вызов признается некорректным.
\item После того, как участники боя зададут все свои вопросы по
  какому-либо решению, жюри задает свои вопросы.
\item Очки выставляются: от 0 до 12 --- за решение, рассказанное
  отвечающим, от 0 до 6 --- за решение, рассказанное оппонентом или
  рецензентом, от 0 до 6 --- за вопросы и замечания (6 очков даются за
  демонстрацию неверности решения). Общее количество очков, набранное
  командами и жюри, равно 12.
\item Команда, обладающая правом вызова, может не вызывать другую
  команду, а отказаться от ведения боя. При этом право вызова
  переходит к третьей команде, и физбой продолжается по правилам для
  двух команд. Отказавшаяся от ведения боя команда имеет право
  выставлять рецензентов, но не получает очков за их деятельность.
\item Каждый участник боя имеет право выходить к доске в качестве
  отвечающего или оппонента не более заранее определенного количества
  раз.
\item Если физбой заканчивается вничью, но необходимо определить
  победителя, командам выдается одна дополнительная
  задача. Определение победителя проходит по правилам конкурса
  капитанов, с той разницей, что задачу может решать вся команда.
\end{enumerate}


\end{document}
