\documentclass[12pt]{article}
\synctex=1
% math symbols
\usepackage{amssymb,amsmath}
% for different compilers
\usepackage{ifpdf}
% geometry of page
\usepackage[left=1.8cm,right=1.8cm,top=0.5cm,bottom=0.5cm]{geometry}
% float pictures
\usepackage{wrapfig}

% if pdflatex, then
\ifpdf
 \usepackage[english,russian]{babel}
 \usepackage[utf8]{inputenc}
 \usepackage[unicode]{hyperref}
 \usepackage[pdftex]{graphicx}
 \usepackage{cmlgc}
% if xelatex, then
\else
% math fonts
 \usepackage{fouriernc}
% xelatex specific packages
 \usepackage[xetex]{hyperref}
 \usepackage{xunicode}	% some extra unicode support
 \usepackage{xltxtra}	% \XeLaTeX macro
 \defaultfontfeatures{Mapping=tex-text}
 \usepackage{polyglossia}	% instead of babel in xelatex
 \setdefaultlanguage{russian}
 % fonts
 \newfontfamily\cyrillicfont{JournalC}
% \newfontfamily\cyrillicfont{PetersburgC}
 %\newfontfamily\cyrillicfont{SchoolBookC}
 \newfontfamily\cyrillicfontsf{TextBookC}
 \setmonofont{Consolas}
\fi

% several pictures in one figure
\usepackage{subfig}
% calc in TeX expressions
\usepackage{calc}
% nice pictures and plots
\usepackage{pgfplots,tikz,circuitikz}
% different libraries for pictures
\usetikzlibrary{%
  arrows,%
  calc,%
  patterns,%
  decorations.pathreplacing,%
  decorations.pathmorphing,%
  decorations.markings%
}
\tikzset{>=latex}

% colors of the hyperlinks
\hypersetup{colorlinks,%
  citecolor=blue,%
  urlcolor=blue,%
  linkcolor=red
}

\tolerance=1000
\emergencystretch=0.74cm

% счётчик задач
\newcounter{notask}
\setcounter{notask}{1}

% условие без картинки
\newcommand{\task}[1]{
  \hrule
  \hbox to \textwidth {%
    \vrule
    \parbox[t]{0.04\textwidth}{\smallskip \centering \arabic{notask}}%
    \vrule%
    \hfill%
    \parbox[t]{0.93\textwidth}{\smallskip #1 \smallskip}\hfill%
    \vrule
  }
  \hrule
  \addtocounter{notask}{1}
  \pagebreak[2]
}

\newlength{\h}
\newsavebox{\taskbox}
\newlength{\x}
\newsavebox{\pictbox}

% условие с картинкой (картинка выравнивается по центру)
\newcommand{\taskpic}[2]{
  \savebox{\taskbox}{\parbox[t]{0.93\textwidth-4.3cm}{\smallskip #1 \smallskip}}
  \savebox{\pictbox}{\parbox[t]{4cm}{\smallskip \centering
      \vspace{0pt} #2 \smallskip}}
  \h=\ht\taskbox
  \advance\h\dp\taskbox
  \x=\ht\pictbox
  \advance\x\dp\pictbox
  \hrule
  \hbox to \textwidth {%
    \vrule\parbox[t][\maxof{\h}{\x}][t]{0.04\textwidth}{ \smallskip
      \centering \arabic{notask} }\vrule%
    \hfill\parbox[t][\maxof{\h}{\x}][t]{0.93\textwidth-4.3cm}{\smallskip #1
        \smallskip}\hfill\vrule%
    \hfill\parbox[t][\maxof{\h}{\x}][c]{4cm}{\hfil #2 \hfil}\hfill\vrule
  }
  \hrule
  \addtocounter{notask}{1}
  \pagebreak[2]
}

\newcommand{\com}[1]{{\Large{\texttt{{\color{red}(#1)}}}}}
% русские единицы измерения в формулах
\newcommand{\unit}[1]{\text{ #1}}
\newcommand{\eps}{\varepsilon}
% не писать номер рисунка в маленьких картинках
\renewcommand{\thesubfigure}{\relax}

% настройки для tikz
% interface = лохматый отрезок
% spring = пружинка
% >=latex --- тип стрелочки
\tikzset{>=latex,%
  interface/.style={postaction={draw,decorate,decoration={border,angle=45,amplitude=0.2cm,segment
        length=1.4323mm}}},%
  spring/.style={decorate,decoration={snake,amplitude=1mm, segment length=2mm},thick}}

\pagestyle{empty}


\begin{document}


\task{ В комнате на столе в патронах стоят 3~лампочки. Снаружи у 
двери комнаты имеются три выключателя, каждым из которых можно 
включить только одну лампочку. Определите каким ключом 
включается каждая лампочка, если открыть дверь и войти в комнату 
можно только один раз. Опишите свои действия. }

\taskpic{ Груз весом $P$ подвешен на невесомом шарнире с тремя 
звеньями. Определите силу натяжения нити, соединяющей точки 
шарнира А и В. }{
\begin{tikzpicture}
  \draw[interface,thick] (1.25,3.5) -- (2.75,3.5);
  \draw[thick] (2,3.5) -- (2,3);
  \begin{scope}
    \draw[thick] (1.75,3.25) -- ++(0.25,0.25) -- ++(0.25,-0.25);
    \draw[thick] (1.75,3.25) -- ++(0.25,-0.25) -- ++(0.25,0.25);
  \end{scope}
  \begin{scope}[yshift=-0.5cm]
    \draw[thick] (1.75,3.25) -- ++(0.25,0.25) -- ++(0.25,-0.25);
    \draw[thick] (1.75,3.25) -- ++(0.25,-0.25) -- ++(0.25,0.25);
  \end{scope}
  \begin{scope}[yshift=-1cm]
    \draw[thick] (1.75,3.25) -- ++(0.25,0.25) -- ++(0.25,-0.25);
    \draw[thick] (1.75,3.25) -- ++(0.25,-0.25) -- ++(0.25,0.25);
  \end{scope}
  \draw (2,2) -- (2,1.75);
  \draw[fill=black] (1.75,1.75) rectangle ++(0.5,-0.5);
  \draw[blue,->] (2.6,3) node[right,black] {\tiny{$A$}} to[out=180,in=-20] (2.15,3.45);
  \draw[blue,->] (2.6,2.5) node[right,black] {\tiny{$B$}} to[out=180,in=-20] (2.15,2.95);
\end{tikzpicture}  
}


\vspace{1cm}

\task{ Сколько времени потребуется для превращения 2~л воды, взятой
  при температуре 20$^\circ$C в пар с температурой 100$^\circ$C?
  Нагревание происходит на горелке, расходующей 0{,}69~кг керосина в
  час.  Теплоемкостью сосуда, в котором находится вода, пренебречь.
  Считать, что все тепло при сгорании керосина подводится к воде.
  Удельная теплота сгорания керосина $q = 4{,}6\cdot 10^7$~Дж/кг,
  удельная теплоемкость воды $c=4{,}2\cdot
  10^3$~Дж/(кг$\cdot^\circ$C), удельная теплота парообразования воды
  $L=2{,}3\cdot10^6$~Дж/кг. }

\taskpic{ Найдите наибольший объем легкой оболочки гелиевого 
метеорологического зонда, который может быть удержан невесомым 
нерастяжимым тросом, прикрепленным к двум одинаковым легким 
пластинам площадью 0{,}07~м$^2$, плотно притертым друг к другу. Нижняя 
пластина жестко прикреплена к земле. Плотность гелия равна 
0{,}178~кг/м$^3$, плотность воздуха --- 1{,}293~кг/м$^3$, атмосферное
давление принять равным 10$^5$~Па. }{
\begin{tikzpicture}
  \draw[line width=0.1cm] (1,0) -- ++(2,0);
  \draw[line width=0.1cm] (1,0.12) -- ++(2,0);
  \draw[thick] (2,0.1) -- (2,1.5);
  \draw[shading=ball] (2,2) circle (0.5);
\end{tikzpicture}  
}


\vspace{1cm}

\task{ По прямому участку железнодорожного пути движется вагон со 
скоростью 14{,}4~км/ч. В вагоне мальчик пускает игрушечный состав по 
рельсам, расположенным поперек вагона. Скорость состава 
относительно пола вагона равна 3~м/с. Найти скорость игрушечного 
состава относительно Земли. }

\task{ Том вплотную подобрался к Джерри, двигаясь с постоянной 
скоростью $v$. В этот момент Джерри начинает убегать от Тома, 
двигаясь по прямой со скоростью $u$ = $k/R$, где $R$ --- расстояние между 
котом и мышью, $k$ --- постоянный независимый коэффициент. Найти 
установившееся расстояние между ними. }

\vspace{1cm}

\newpage\null\thispagestyle{empty}\newpage

\taskpic{ Из четырех нихромовых проволочек с удельным 
сопротивлением $\rho$, площадью сечения $S$, длиной $L$ каждая, 
выполнена фигура, представляющая собой крест. Крест подключают к 
источнику постоянного тока напряжением $U$, как показано на 
рисунке (положительный полюс к точке пересечения проволочек, 
отрицательный полюс к концам креста). Фигуру помещают в термос с 
дистиллированной водой. В некоторый момент времени  замыкают 
ключ. Вода закипела через время $\Delta t$. Сколько воды находилось в 
термосе? Удельная теплоемкость воды $c$, начальная температура $T$.  
Потерями тепла пренебречь. }{
\begin{tikzpicture}
  \draw[blue,thick] (0.75,1.5) to[out=45] (2,2.5);
  \draw[red,thick] (0,1.5) to[out=90,in=90] (2.5,2.5);
  \draw[red,thick] (0.75,2.25) to[out=90] (2.5,2.5);
  \draw[red,thick] (1.5,1.5) to[out=20,in=-100] (2.5,2.5);
  \draw[red,thick] (0.75,0.75) to[out=-20,in=-100] (2.5,2.5);
  \draw[line width=0.15cm] (0,1.5) -- (1.5,1.5);
  \draw[line width=0.15cm] (0.75,0.75) -- (0.75,2.25);
  \draw[red,fill=white] (2.5,2.5) circle (0.05) node[below right] {$-$};
  \draw[blue,fill=white] (2,2.5) circle (0.05) node[below] {$+$};
\end{tikzpicture}  
}

\task{ На поверхности озера Байкал зимой намерзает толстый слой 
льда. Предположим, что где-то в декабре толщина льда составляет $x$ 
= 80~см. Температура воздуха $t$ = -40$^\circ$С. С какой скоростью (в~мм/ч) 
увеличивается в этот период толщина слоя льда?

Для льда: плотность $\rho$ = 0{,}92~г/см$^3$, удельная теплота
плавления $\lambda = 3{,}3\cdot10^5$~Дж/кг, коэффициент
теплопроводности $k$ = 2{,}2 Вт/(м$\cdot^\circ$C). Количество теплоты,
проходящее в единицу времени через слой вещества площадью $S$ и
толщиной $h$ при разности температур $\Delta t$ между поверхностями,
определяется соотношением $q = kS\Delta t/h$. }

% Россия 9 класс, сложно, но не безумно


\vspace{1cm}


\task{ Имеются две трубы, подсоединенные к смесителю. На каждой из 
труб имеется кран, которым можно регулировать поток воды по 
трубе, изменяя его от нуля до максимального значения~1 л/с. В 
трубах течет вода с температурами $t_1$ = 20$^\circ$C и $t_2$ = 60$^\circ$C. Из 
смесителя вытекает вода, температура которой во всех точках 
одинакова. Постройте график зависимости максимального потока 
воды, вытекающего из смесителя, от температуры этой воды. }

\task{ Отбросив с помощью зеркальца на близкую поверхность 
солнечный зайчик, наблюдатель затем расположил параллельно 
зеркальцу на малом расстоянии от него карандаш. Как при этом 
изменится вид солнечного зайчика? }

% Оптика

\vspace{1cm}

\taskpic{ Две пружины с коэффициентами жесткости 12~Н/м и 8~Н/м  и легкая 
шайба, скользящая вдоль стержня без трения,  соединены вместе, 
как показано на рисунке. К свободному концу пружины прикладывают 
такую силу $F$, что он движется вправо с постоянной скоростью 
0{,}1~м/с. Найдите скорость шайбы. Постройте график зависимости 
прикладываемой силы $F$ от времени. }{
\begin{tikzpicture}
  \draw[very thick,interface] (0,0.5) -- (0,3.5);
  \draw[spring,blue] (0,2) -- (1.5,2) node[midway,below] {$k_1$};
  \draw[thick,pattern=north east lines] (1.5,2.5) rectangle
  ++(0.25,-1);
  \draw[spring,red] (1.75,2) -- ++(1,0) node[midway,below] {$k_2$};
  \draw[->] (2.75,2) -- ++(0.5,0) node[above] {$v$};
\end{tikzpicture}  
}

\taskpic{ Два одинаковых проводника, изготовленных так, что их 
удельное сопротивление линейно изменяется с расстоянием от 
начала проводника: $\rho = kL$, где $\rho$ --- удельное сопротивление 
проводника, $k$ --- известный постоянный коэффициент,  $L$ --- 
расстояние от начала проводника до данной точки, соединены 
параллельно так, что у одного удельное сопротивление возрастает 
справа налево, а у другого наоборот --- слева направо. Эта схема 
подключена к источнику постоянного тока с напряжением $U_0$ 
(см.рис.). Какое напряжение показывает идеальный вольтметр, 
соединяющий середины этих проводников? }{
\begin{circuitikz}
  \draw[o-,thick] (1.5,0.5) -- (0.5,0.5) -- (0.5,3.5);
  \draw[thick] (0.5,3.5) to[generic] (3.5,3.5) -- (3.5,1.5)
  to[generic] (0.5,1.5);
  \draw[o-,thick] (2.5,0.5) -- (3.5,0.5) -- (3.5,1.5);
  \draw[thick] (2,3.25) to[voltmeter] (2,1.75);
  \draw (2,0.5) node {$U_0$};
\end{circuitikz}
}


\newpage\null\thispagestyle{empty}\newpage

\task{ Два кота загнали мышку в узкий коридор и с двух сторон 
приближаются к ней, а мышка бегает между ними. Сколько раз мышка 
успеет добежать от одного кота до другого,  если скорость ее 
движения 2~м/с, а коты <<наступают>> со скоростью 0{,}5~м/с? Мышка 
поворачивает каждый раз на расстоянии 0{,}5~м от кота, не тратя на 
это времени. Мышка прекратит сопротивление, когда расстояние 
между котами будет равно 1{,}5~м. Известно, что длина коридора 10~м, 
коты начали двигаться одновременно и мышка в начальный момент 
была на расстоянии 0{,}5~м от одного из котов. }

\task{ Тело отпускают на высоте 15~м над стальной плитой. Удары тела о 
плиту абсолютно упругие. Постройте графики зависимости скорости 
тела и пути, пройденного телом, от времени за первые 6~секунд 
движения. }

\vspace{1cm}

\task{ В сосуде у поверхности воды плавает кусок льда с вмерзшей в 
него медной дробинкой массой 3~г. Сосуду сообщили 24~кДж теплоты, и 
дробинка утонула. Кусок льда какой массы плавал у поверхности?
Температура воды и льда 0$^\circ$С. Удельная теплота плавления льда 
$\lambda_{\mbox{\textit{в}}}$ = 340~кДж/кг, плотность воды $\rho_{\mbox{\textit{в}}}$ = 
1000~кг/м$^3$, плотность льда $\rho_{\mbox{\textit{л}}}$ = 900~кг/м$^3$, плотность 
меди $\rho_{\mbox{\textit{м}}}$ = 8900~кг/м$^3$. }

\task{ На какой глубине должна находиться опора, чтобы цилиндр 
высотой 24~см, плотно стоящий на ней, не всплывал? Плотность 
материала цилиндра в два раза меньше плотности воды, а площадь 
сечения в сто раз больше площади опоры. Атмосферное давление 
100~кПа, плотность воды 1000~кг/м$^3$. Соприкасающиеся поверхности 
цилиндра и опоры считать абсолютно гладкими. Ускорение 
свободного падения принять равным 10~м/с$^2$. }

\vspace{1cm}

\taskpic{ Двое друзей рассказывали, как здорово они умеют ездить на 
велосипеде. Первый говорит: <<Однажды я так сумел проехать, что 
график зависимости скорости от времени представлял точную 
полуокружность>>. <<А я умудрился так проехать, --- говорит другой, -- 
что этот график представлял равнобедренный треугольник>>. 
Определите, какое расстояние проехал каждый из них за 10~секунд, 
рассчитайте их средние скорости и определите, кто из них приврал. 
}{
\begin{tikzpicture}
  \draw[blue,very thick] (3.5,0) arc (0:180:3.5/2);
  \draw[red,very thick] (0,0) -- (1.75,1.75) -- (3.5,0);
  \draw[thick,->] (0,0) -- (3.8,0);
  \draw[thick,->] (0,0) -- (0,2.5) node[right] {\tiny{$v,\unit{м/c}$}};
  \draw[red] (2.5,0.7) node {$2$};
  \draw[blue] (2.5,1.9) node {$1$};
  \foreach \x in {2,4,6,8,10} {
    \draw (0.35*\x,0.1) -- ++(0,-0.2) node[below] {\tiny{\x}} ;
  }
  \foreach \y in {1,2,3,4,5} {
    \draw (0.1,0.35*\y) -- ++(-0.1,0) node[left=-2] {\tiny{\y}} ;
  }
  \draw (3.5/2,-0.5) node{\tiny{$t,\unit{с}$}};
\end{tikzpicture}  
}

\task{ Во сколько раз будет отличаться минимальная начальная 
скорость, необходимая для того, чтобы перепрыгнуть яму, от 
минимальной начальной скорости, с которой можно перебраться 
через ту же яму с помощью жесткого легкого шеста, опирающегося на 
центр дна ямы? Глубина ямы $H$, ширина $L$. }

% Район 10 класса. 

\newpage\null\thispagestyle{empty}\newpage

\taskpic{ По горизонтальной плоскости скользит без трения точечная 
шайба массы $m$. Скорость шайбы $v$. Перпендикулярно направлению 
движения шайбы движется лента транспортера с такой же по модулю 
скоростью $v$. Ширина транспортера $H$. Какой должна быть сила 
трения между поверхностями шайбы и транспортера для того, чтобы 
шайба переехала через него? }{
\begin{tikzpicture}
  \draw[thick] (0,3) rectangle ++(3,-1);
  \draw[thick,->] (2.5,2.5) -- (3.4,2.5) node[below] {$v$};
  \draw[thick,->] (1.3,2) -- (1.3,2.75) node [right] {$v$};
  \draw[fill=black] (1.3,2) circle (0.05);
  \draw[blue,thick,<->] (0.25,2) -- (0.25,3) node [midway,right] {$H$};
\end{tikzpicture}  
}

% Район 9 класса. Трение + переход в СО + работа. 

\task{ Горит башня, причем возгорание произошло в двух местах: 
первое --- на 1/10 высоты башни, а второе на $L$ = 220~м выше. Пламя 
распространяется вверх в 7 раз быстрее, чем вниз. Башня сгорела 
дотла за $t_1$ = 60~ч. Если бы $L$ было в 2~раза больше, башня бы сгорела 
за $t_2$ = 61~ч, а если бы в 2~раза меньше, то время бы не изменилось 
(60~ч). Чему была равна высота башни? }

\vspace{1cm}

\task{ На горизонтальной шероховатой поверхности лежит цепочка из 
$N$ шариков массы $m$ каждый, связанных пружинками жесткости $k$. Все 
пружинки одинаковые и подчиняются закону Гука ($F_{\mbox{\textit{упр}}} = 
-kx$). Длина каждой пружинки в нерастянутом состоянии равна 0. 
Цепочку как-то растянули. Найдите максимально возможную длину 
цепочки, при которой все шарики неподвижны. Коэффициент трения 
между поверхностью и шариками равен $\mu$, ускорение свободного 
падения равно $g$. Размерами шариков пренебречь. }

% Город 9 класса. Трение + арифм. прогрессия. 

\task{ Две дороги пересекаются под углом $\alpha$. По ним к перекрестку 
двигаются два автомобиля. Первый имеет скорость $v_1$, а второй --- 
$v_2$. В некоторый момент времени первый автомобиль находится на 
расстоянии $L_1$ от перекрестка, а второй на расстоянии $L_2$ от 
перекрестка. Найдите минимальное расстояние между автомобилями 
в процессе их движения. }

\vspace{1cm}

\taskpic{ Грузы, имеющие массы $M$ и $m$ ($M > m$), при помощи невесомой 
нерастяжимой нити подвешены на блоке. С каким минимальным 
ускорением нужно двигать блок в вертикальном направлении, чтобы 
ускорения грузов были направлены в одну сторону? Ускорение 
свободного падения $g$. Сопротивлением воздуха пренебречь. 
}{
\begin{tikzpicture}
  \draw[interface,thick] (1,4) -- ++(2,0);
  \draw[thick] (2,4) -- ++(0,-1);
  \draw[very thick] (2,3) circle (0.5);
  \draw[thick] (2.5,3) -- ++(0,-1.25);
  \draw[thick] (1.5,3) -- ++(0,-1);
  \draw[fill=black] (1.25,2) rectangle ++(0.5,-0.5);
  \draw (0.9,1.75) node {$M$};
  \draw[fill=gray] (2.3,1.75) rectangle ++(0.4,-0.5);
  \draw (3,1.5) node {$m$};
\end{tikzpicture}  
}

\task{ Катушку радиуса $R$, находящуюся на горизонтальной 
поверхности, тянут за нить, намотанную на ось катушки радиуса $r$. 
Нить движется со скоростью $v$ в горизонтальном направлении. 
Найдите поступательную скорость оси катушки относительно 
поверхности. Катушка не проскальзывает по поверхности. }

\vspace{1cm}

\task{ С помощью термометра измеряют попеременно температуры 
жидкостей, налитых в два калориметра. Показания термометра: 80, 16, 
78, 19 $^\circ$C. Что покажет термометр после следующего переноса? После 
большого числа переносов? }

\task{ Толстостенная лодка с вертикальными стенками и отверстием в 
дне достаточно долго свободно плавает в озере. Затем отверстие 
снаружи затыкают, и потом внутрь лодки пускают плавать бревно. 
Повысится или понизится после этого уровень воды в лодке 
относительно уровня воды в озере? }

\newpage\null\thispagestyle{empty}\newpage

\task{ В стакан с водой опустили нагреватель и сняли зависимость 
температуры воды $T$ от времени $t$ (см. таблицу). На сколько 
градусов остынет вода за 1~мин, если нагреватель отключили от 
сети при температуре 50$^\circ$C? Закипит ли вода, если нагреватель не 
выключать достаточно долго? Мощность нагревателя считать 
неизменной.

\begin{tabular}{| c | c | c | c | c | c | c | c | c | c | c |}
\hline
$t$,~мин & 0 & 1 & 2 & 3 & 4 & 5 & 6 & 7 & 8 & 9 \\
\hline
$T$,~$^\circ$C & 20 & 26,2 & 31,8 & 36,8 & 41,4 & 45,6 & 49,3 & 52,7 & 55,8 & 61,1 \\
\hline
\end{tabular} }

\task{ Вес $P$ системы, состоящей из стакана с водой и пробкового 
шарика, измеряют в следующих пяти случаях:
\begin{enumerate}
\item шарик свободно плавает в стакане (показания весов $P_1$);
\item шарик лежит на чашке весов рядом со стаканом ($P_2$);
\item шарик удерживается в полностью погруженном состоянии тонкой 
невесомой нитью, прикрепленной к дну стакана ($P_3$);
\item шарик удерживается в полностью погруженном состоянии тонкой 
невесомой спицей, закрепленной над стаканом ($P_4$);
\item шарик свободно всплывает ($P_5$).
\end{enumerate}
Расставить показания весов в порядке их возрастания. }

\vspace{1cm}

\task{ Две невесомые пружины имеют длины $l_1$, $l_2$ и жесткости $k_1$, $k_2$ 
соответственно. Одна пружина вставлена в другую. Концы пружин 
попарно скреплены. Другими точками пружины друг друга не 
касаются. Какова жесткость получившейся пружины? }

\task{ Моток голой проволоки, содержащей семь с половиной витков, 
растянут на двух вбитых в доску гвоздях, к которым присоединены 
его концы. Подключив к гвоздям приборы, измерили сопротивление 
цепи между гвоздями. Во сколько раз изменится это сопротивление, 
если моток снять с доски и размотать, оставив концы 
присоединенными к гвоздям? }

\newpage\null\thispagestyle{empty}\newpage

\task{ Три брата вместе выехали на конях из дворца и поехали к Кощею 
Бессмертному. Братья ехали по одной дороге, скорость каждого из 
них была постоянна. Скорость среднего брата равнялась 24~км/ч, 
скорость младшего брата --- 20~км/ч. Первым к Кощею приехал старший 
брат, спустя 1~час к Кощею приехал средний брат. Через 1~час после 
приезда среднего брата к Бессмертному приехал младший брат. 
Найдите скорость старшего брата. }

\task{ Цилиндрический медный проводник площадью поперечного 
сечения $S$ = 0{,}1~см$^2$ подключается к источнику постоянного тока. 
Температура проводника начинает увеличиваться. Как видно из 
графика зависимости температуры от времени, через время $\tau_1$ = 
10~мин температура проводника становится равной $t_1$ = 90$^\circ$C.
\begin{enumerate}
\item За какое время температура проводника достигла бы значения 
$t_1$, если бы проводник был окружен теплонепроницаемой оболочкой?
\item Найдите силу тока в проводнике.
\item Предположим, что по истечении времени $\tau_2$ = 5~мин проводник 
был отключен от источника тока и начал остывать. Определите, за 
какое приблизительно время $\Delta\tau$ температура проводника 
изменится от 70$^\circ$C до 65$^\circ$C.
\end{enumerate}
Для меди: удельная теплоемкость $c$ = 390~Дж/(кг$\cdot^\circ$C), удельное 
сопротивление $\rho_{\mbox{\textit{м}}}$ = 1{,}75$\cdot 10^{-8}$~Ом$\cdot$м, плотность $\rho$ 
= 8{,}9$\cdot10^3$~кг/м$^3$. }

\begin{figure}[h]
  \centering
  \begin{tikzpicture}
    \draw[help lines,step=0.2] (0,0) grid (10.5,9.25);  
    \draw [very thick,->] (0,0) -- (11,0);
    \draw [very thick,->] (0,0) -- (0,9.5) node[right] {$t,{}^\circ C$};
    \draw [line width=2.5,red] (0,2) .. controls (2,6) and (6,7.75)
    .. (10,9);
    \draw[thick,blue,dashed] (0,7) -- (10.5,7);
    \draw[thick,blue,dashed] (0,9) -- (10.5,9);
    \draw[thick,blue,dashed] (5,0) -- (5,9.25);
    \draw[thick,blue,dashed] (10,0) -- (10,9.25);
    \draw[thick,blue,dashed] (3,0) -- (3,9.25);
    \foreach \x in {1,2,3,4,5,6,7,8,9,10} {
      \draw (\x,0.1) -- ++(0,-0.2) node[below] {\x} ;
    }
    \foreach \y in {10,20,30,40,50,60,70,80,90} {
      \draw (0.1,0.1*\y) -- ++(-0.2,0) node[left] {\y} ;
    }
    \draw (3,-1) node {$\tau_0$};
    \draw (5,-1) node {$\tau_2$};
    \draw (10,-1) node {$\tau_1$};
    \draw (5.5,-1.7) node {$\tau,\unit{мин}$};
  \end{tikzpicture}  
\end{figure}

% Тоже Россия, довольно сложно

\newpage\null\thispagestyle{empty}\newpage

\task{ Два автомобиля находятся на шоссе на расстоянии 5~км друг от 
друга. По правилам гонки они обязаны все время двигаться с 
ускорением 1~м/с$^2$ относительно земли, причем направление 
ускорения каждый из них может менять в любой момент времени и 
неограниченное число раз. Гонка считается завершенной, когда 
автомобили оказываются рядом друг с другом и их скорости 
относительно друг друга в этот момент равны нулю. Найдите 
минимальное возможное время от начала гонки до ее завершения. }

% Город 9 класса. 

\task{ Из муравейника за гусеницей, расстояние до которой 1~м, 
выползает группа муравьев. Все муравьи движутся с постоянными 
скоростями, которые у разных особей разные и меняются от 1~см/с до 
2~см/с. Через 30~с муравей Ферда, который до этого двигался со 
скоростью 1~см/с, начинает двигаться с переменной скоростью, 
причем его скорость всегда в два раза выше, чем скорость 
окружающих его в данный момент муравьев. Успеет ли Ферда первым 
прибежать к гусенице?  Считайте, что характер движения других 
муравьев при этом не меняется. }

% Город 9, равноускоренное движение, сложно

\end{document}
