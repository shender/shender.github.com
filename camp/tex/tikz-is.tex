% основные библиотеки
\usepackage{pgfplots,tikz}
\usetikzlibrary{%
  arrows,%
  calc,%
  patterns,%
  intersections,%
  decorations.pathreplacing,%
  decorations.pathmorphing,%
  decorations.text,%
  decorations.markings%
}


% основные используемые стили
% стиль для стрелки
\tikzset{>=latex,%
% платформа: пол или потолок  
  interface/.style={postaction={draw,decorate,decoration={border,
        angle=45, amplitude=0.2cm, segment length=1mm}}},%
% пружина  
  spring/.style={decorate,decoration={coil,amplitude=1mm, segment
      length=1mm},thick},%
% стрелка в середине отрезка  
  marrow/.style={postaction={draw,decorate,decoration={markings,
        mark=at position 0.6 with {\arrow{latex}}}}}}

% обозначение угла
\tikzset{arcnode/.style={
            decoration={
                        markings, raise = 2mm,  
                        mark=at position 0.5 with { 
                                    \node[inner sep=0] {#1};
                        }
            },
            postaction={decorate}
      }
}

% команда для отметки угла: проводит дугу между двумя лучами,
% проведёнными между точками #2--#3 и #2--#4
% первый аргумент - необязательный, стиль линии
% #2,#3,#4 - точки
% #5 - радиус дуги для обозначения угла
% #6 - обозначение угла, например, $\alpha$
\newcommand*\marktheangle[6][]{
            \draw[thick,arcnode={#6},#1] let \p2=($(#3)-(#2)$),%
                        \p3=($(#4)-(#2)$),%
                        \n2 = {atan2(\x2,\y2)},%
                        \n3 = {atan2(\x3,\y3)}%
                        in ($(\n2:#5)+(#2)$) arc (\n2:\n3:#5);
            }

% для работы с графиками            
\pgfplotsset{compat=newest}

% библиотеки для электричества
\usetikzlibrary{circuits.ee,circuits.ee.IEC}

% амперметр
\tikzset{circuit declare symbol = ammeter}
\tikzset{set ammeter graphic ={draw,generic circle IEC, minimum
    size=5mm,info=center:A}} 

% вольтметр
\tikzset{circuit declare symbol = voltmeter}
\tikzset{set ammeter graphic ={draw,generic circle IEC, minimum
    size=5mm,info=center:V}} 
