% счётчик задач
\newcounter{notask}
\setcounter{notask}{1}

% \task{УСЛОВИЕ ЗАДАЧИ}
% задача без картинки
% оформлена как таблица с двумя колонками
% ширина первой колонки (номер столбца) фиксирована, 0.3cm
% ширина второй колонки автоматически рассчитывается из ширины
% страницы (с учётом всевозможных отступов)
% для удобства восприятия вначале текста вставляется вертикальный
% зазор в 1mm
\newcommand*{\task}[1]{%
  \begin{tabular}{|p{0.3cm}|m{\textwidth-0.95cm}|}
    \cline{1-2}
    \centering \arabic{notask} & { \vspace{1mm} #1 } \\
    \cline{1-2}
  \end{tabular}
  
  \vspace{-1pt}

  \addtocounter{notask}{1}
}

% \taskpic[ШИРИНА КАРТИНКИ]{УСЛОВИЕ ЗАДАЧИ}{КАРТИНКА}
% задача с картинкой
% оформлена как таблица с тремя колонками
% первый аргумент - необязательный, по умолчанию ширина картинки равна
% 4cm, но можно выставить свою
% ширина второй колонки (условие задачи) рассчитывается из ширины
% страницы и ширины картинки
\newcommand*{\taskpic}[3][4cm]{%
  \begin{tabular}{|p{0.3cm}|m{\textwidth-1.385cm-#1}|m{#1}|}
    \cline{1-3}
    \centering \arabic{notask} & { \vspace{1mm} #2 } & { \parbox{#1}{
        \vspace{1mm} \centering #3 \vspace{1mm}}}\\ 
    \cline{1-3}
  \end{tabular}

  \vspace{-1pt}

  \addtocounter{notask}{1}
}

