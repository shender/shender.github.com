% Описание протоколов физбоя.
% оформляется с помощью tikz, так что необходимо указывать
\usepackage{tikz}

% переменные и счётчики

\newcounter{oldN}
\setcounter{oldN}{0}

\newcounter{newN}
\setcounter{newN}{0}

% баллы жюри
\newcounter{jury}
\setcounter{jury}{12}

% двойной физбой: заголовок протокола
% \pdtitle{название команды 1}{название команды 2}
\newcommand{\pdtitle}[2]{%
  \tikzset{node distance=0pt,minimum height=1cm,>=angle 90};  
  \tikzset{call/.style={very thick,black!60,decoration={random
        steps,segment length=5pt,amplitude=0.5pt},decorate}} 
  \node[draw,thick, minimum width=0.1\textwidth,
  rectangle] (n0) {\large{\textbf{№}}}; 
  \node[draw,thick, minimum width=0.35\textwidth,
  rectangle, right=of n0] (t1-0) {\large{\textbf{#1}}}; 
  \node[draw,thick, minimum width=0.35\textwidth,
  rectangle, right=of t1-0] (t2-0) {\large{\textbf{#2}}}; 
  \node[draw,thick, minimum width=0.2\textwidth,
  rectangle, right=of t2-0] (j0) {\large{\textbf{Жюри}}};
}

% двойной физбой: разбор задачи
% \pdcall{направление вызова}{номер
% задачи}{баллы1}{баллы2}{фио1}{фио2};
% условные обозначения для направления вызова:
% -> --- команда 1 вызывает команду 2
% <- --- команда 2 вызывает команду 1
% ->| --- команда 1 вызывала команду 2, но получила проверку корректности
% |<- --- команда 2 вызывала команду 1, но получила проверку корректности
\newcommand{\pdcall}[6]{%
  \addtocounter{newN}{1};
  % баллы жюри
  \addtocounter{jury}{-#3}
  \addtocounter{jury}{-#4}
  \def\a{t1-\arabic{newN}};
  \def\b{t2-\arabic{newN}};
  \tikzset{minimum width=1cm};
  % расположение всех баллов
  \node[below=of n\arabic{oldN},yshift=-0.5cm] (n\arabic{newN}) {#2};
  \node[below=of t1-\arabic{oldN},yshift=-0.5cm] (\a)
  {\Large{#3}}; 
  \node[below=of t2-\arabic{oldN},yshift=-0.5cm] (\b)
  {\Large{#4}}; 
  \node[below=of j\arabic{oldN},yshift=-0.5cm] (j\arabic{newN})
  {\Large{\arabic{jury}}};
  % инициалы отвечающего и оппонента
  \node[blue,above=of \a,yshift=-0.4cm] {#5};
  \node[blue,above=of \b,yshift=-0.4cm] {#6};
  % рисование стрелки
  \def\calllr{->}
  \def\callrl{<-}
  \def\calllor{->|}
  \def\callrol{|<-}
  \begingroup
  \def\tmp{#1}%
  \ifx\tmp\calllr
    \draw[call,*->] (\a.east) -- (\b.west); 
  \else\ifx\tmp\callrl
    \draw[call,<-*] (\a.east) -- (\b.west); 
  \else\ifx\tmp\calllor
    \draw[call,*->,rounded corners=0.1cm]
    (\a.20) -- (\b.160) -- (\b.200) -- (\a.340);
  \else\ifx\tmp\callrol
    \draw[call,*->,rounded corners=0.1cm]
    (\b.160) -- (\a.20) -- (\a.340) -- (\b.200);
  \fi\fi\fi\fi
  \endgroup
  \setcounter{jury}{12}
  \addtocounter{oldN}{1};
}

% двойной физбой: подведение итогов
% \pdsum{сумма 1}{сумма 2}{сумма жюри};
\newcommand{\pdsum}[3]{%
  % расстановка баллов
  \node[below=of n\arabic{oldN},yshift=-0.5cm] (sum) {Итог:};
  \node[below=of t1-\arabic{oldN},yshift=-0.5cm] (t1sum)
  {\LARGE{#1}}; 
  \node[below=of t2-\arabic{oldN},yshift=-0.5cm] (t2sum)
  {\LARGE{#2}}; 
  \node[below=of j\arabic{oldN},yshift=-0.5cm] (jsum)
  {#3};
  % рамка
  \draw[very thick] (n0.north west) -- (sum.north west -| n0.west) --
  (sum.north west -| j0.east) -- (j0.north east);
  \draw[very thick] (n0.north east) -- (n0.north east |- sum.north);
  \draw[very thick] (j0.north west) -- (j0.north west |- sum.north);
  % пунктир
  \draw[gray,loosely dashed] (t1-0.north east) -- (t1-0.east |-
  t1sum.north);
  % установка счётчиков в исходное состояние
  \setcounter{oldN}{0}
  \setcounter{newN}{0}
  \setcounter{jury}{12}
}

\newcommand{\pttitle}[3]{%
  \tikzset{node distance=0pt,minimum height=1cm,>=angle 90};
  \node[draw,thick, minimum width=0.1\textwidth,
  rectangle] (n0) {\large{\textbf{№}}}; 
  \node[draw,thick, minimum width=0.25\textwidth,
  rectangle, right=of n0] (t1-0) {\large{\textbf{#1}}}; 
  \node[draw,thick, minimum width=0.25\textwidth,
  rectangle, right=of t1-0] (t2-0) {\large{\textbf{#2}}};
  \node[draw,thick, minimum width=0.25\textwidth,
  rectangle, right=of t2-0] (t3-0) {\large{\textbf{#3}}}; 
  \node[draw,thick, minimum width=0.15\textwidth,
  rectangle, right=of t3-0] (j0) {\large{\textbf{Жюри}}};
}

% тройной физбой: вызов
% \ptcall{#1}{номер задачи}{баллы1}{баллы2}{баллы3}{участник 1}{участник 2}{участник 3}
% здесь уже не надо указывать направление вызова, поскольку в тройном
% физбое порядок определён после конкурса капитанов
% первый аргумент (#1) показывает, есть ли проверка корректности
% если вызов обычный - аргумент пустой
% если проверка - можно написать всё, что угодно (например,
% какую-нибудь букву)
\newcommand{\ptcall}[8]{
  \tikzset{call/.style={very thick,black!60,decoration={random
        steps,segment length=5pt,amplitude=0.5pt},decorate}} 
  \tikzset{minimum width=1cm};
  \addtocounter{newN}{1};
  \def\a{t1-\arabic{newN}};
  \def\b{t2-\arabic{newN}};
  \def\c{t3-\arabic{newN}};
  \edef\cl{\arabic{newN}};  
  % баллы жюри
  \addtocounter{jury}{-#3}
  \addtocounter{jury}{-#4}
  \addtocounter{jury}{-#5} 
  % расположение всех баллов
  \node[below=of n\arabic{oldN},yshift=-0.6cm] (n\arabic{newN}) {#2};
  \node[below=of t1-\arabic{oldN},yshift=-0.6cm] (\a)
  {\Large{#3}}; 
  \node[below=of t2-\arabic{oldN},yshift=-0.6cm] (\b)
  {\Large{#4}};
  \node[below=of t3-\arabic{oldN},yshift=-0.6cm] (\c)
  {\Large{#5}}; 
  \node[below=of j\arabic{oldN},yshift=-0.6cm] (j\arabic{newN})
  {{\arabic{jury}}};
  % инициалы слева направо
  \node[blue,above=of \a,yshift=-0.4cm] {#6};
  \node[blue,above=of \b,yshift=-0.4cm] {#7};
  \node[blue,above=of \c,yshift=-0.4cm] {#8};
  % поскольку порядок вызовов определяется автоматически,
  % здесь сразу рисуется стрелка вызова в зависимости от номера хода
  % кроме того, проверяем, есть ли проверка корректности
  % 1-й ход
  \ifnumequal{\cl}{1}{
    \ifblank{#1}
    {
      \draw[call,*->] (\a.east) -- (\b.west);
    }
    {
      \draw[call,*->,rounded corners=0.1cm] (\a.20) -- (\b.160) --
      (\b.200) -- (\a.340); 
    }
  }{}
  % 2-й ход
  \ifnumequal{\cl}{2}{
    \ifblank{#1}
    {
      \draw[call,*->] (\b.east) -- (\c.west);
    }
    {
      \draw[call,*->,rounded corners=0.1cm] (\b.20) -- (\c.160)
      -- (\c.200) -- (\b.340);
    }
  }{}
  % 3-й ход
  \ifnumequal{\cl}{3}{
    \ifblank{#1}
    {
      \draw[call,*-] (\c.east) -- (\c.east -| j0.west);
      \draw[call,->] (t1-0.west |- \a.west) -- (\a.west); 
    }
    {
      \draw[call,*-] (\c.20) -- (\c.20 -| j0.west);
      \draw[call,->] (\c.340 -| j0.west) -- (\c.340);
      \draw[call,rounded corners=0.1cm] (t1-0.west |- \a.160) --
      (\a.160) -- (\a.200) -- (t1-0.west |- \a.200);
    }
  }{}
  % 4-й ход
  \ifnumequal{\cl}{4}{
    \ifblank{#1}
    {
      \draw[call,*-] (\a.west) -- (\a.west-| t1-0.west);
      \draw[call,->] (\c.east -| t3-0.east) -- (\c.east);
    }
    {
      \draw[call,*-] (\a.160) -- (\a.160 -| t1-0.west);
      \draw[call,->] (\a.200 -| t1-0.west) -- (\a.200);
      \draw[call,rounded corners=0.1cm] (\c.20 -| j0.west) --
      (\c.20) -- (\c.340) -- (\c.340 -| j0.west); 
    }
  }{}
  % 5-й ход
  \ifnumequal{\cl}{5}{
    \ifblank{#1}
    {
      \draw[call,*->] (\c.west) -- (\b.east);
    }
    {
      \draw[call,rounded corners=0.1cm,*->] (\c.160) -- (\b.20)
      -- (\b.340) -- (\c.200);
    }
  }{}
  % 6-й ход
  \ifnumequal{\cl}{6}{
    \ifblank{#1}
    {
      \draw[call,*->] (\b.west) -- (\a.east);
    }
    {
      \draw[call,rounded corners=0.1cm,*->] (\b.160) -- (\a.20)
      -- (\a.340) -- (\b.200);
    }
  }{}
  % переход на следующий круг
  \setcounter{jury}{12}
  \addtocounter{oldN}{1}
}

% тройной физбой: подведение итогов
% \ptsum{баллы 1}{баллы 2}{баллы 3}{баллы жюри}
\newcommand{\ptsum}[4]{%
% основная рамка
  \draw[very thick] (t1-0.south west) -- (t1-0.south west |-
  \a.south);
  \draw[very thick] (t3-0.south east) -- (t3-0.south east |-
  \a.south);
  \draw[very thick] (n0.south west |- \a.south) -- (j0.south east |-
  \a.south);
  \draw[very thick] (n0.north west) -- (n0.north west |- \a.south);
  \draw[very thick] (j0.north east) -- (j0.north east |- \a.south); 
% пунктир
  \draw[gray,loosely dashed] (t1-0.south east) -- (t1-0.south east |-
  \a.south); 
  \draw[gray,loosely dashed] (t2-0.south east) -- (t2-0.south east |-
  \a.south); 
% строчка с итоговыми баллами   
  \node[below=of n\arabic{oldN},yshift=-0.6cm] (n-f) {Итог:};
  \node[below=of \a,yshift=-0.6cm] (t1-f) {\LARGE{#1}};
  \node[below=of \b,yshift=-0.6cm] (t2-f) {\LARGE{#2}};
  \node[below=of \c,yshift=-0.6cm] (t3-f) {\LARGE{#3}};
  \node[below=of j\arabic{oldN},yshift=-0.6cm] (j-f) {#4};
% установка счётчиков в исходное состояние
  \setcounter{oldN}{0}
  \setcounter{newN}{0}
  \setcounter{jury}{12}
}

